\documentclass[a4paper,10pt]{article}
\usepackage{ctex}
\usepackage{listings}
\usepackage{xcolor}
\usepackage[margin=0.7cm,top=1.2cm,bottom=1.2cm,nomarginpar]{geometry}
\usepackage[hidelinks]{hyperref}
\usepackage{fancyhdr}

\title{ACM算法竞赛模板}
\author{mouhua}
\date{\today}

% 代码格式设置
\lstset{
    backgroundcolor=\color[RGB]{245,245,245},
    keywordstyle=\color[RGB]{40,40,255},
    commentstyle=\color[RGB]{0,96,96},
    stringstyle=\color[RGB]{128,0,0},
    breaklines=true,
    numberstyle=\tiny\color{gray},
    numbers=left,
    basicstyle=\ttfamily\small,
    frame=single,
    tabsize=4,
    showstringspaces=false,
    language=C++,
    xleftmargin=0pt,
    xrightmargin=0pt,
    keepspaces=true,
    columns=flexible,
    belowskip=-0.1em,
    aboveskip=0.1em,
    linewidth=\textwidth,
    resetmargins=true,
    breakindent=0pt
}

% 设置页眉页脚
\pagestyle{fancy}
\fancyhf{}
\fancyhead[C]{ACM算法竞赛模板}
\fancyfoot[C]{\thepage}
\renewcommand{\headrulewidth}{0.4pt}
\renewcommand{\footrulewidth}{0.4pt}
\setlength{\headheight}{13.6pt}
\setlength{\headsep}{10pt}
\setlength{\footskip}{13pt}

\begin{document}

\maketitle
\thispagestyle{fancy}
\newpage
\tableofcontents
\thispagestyle{fancy}
\newpage

\section{其他}
\thispagestyle{fancy}

\subsection{O3}
\thispagestyle{fancy}

\noindent\begin{lstlisting}
#pragma GCC optimize("O3,unroll-loops")
#pragma GCC target("avx2,avx512f,avx512vl,avx512bw,avx512dq,avx512cd,avx512vbmi,avx512vbmi2,avx512vpopcntdq,avx512bitalg,bmi,bmi2,lzcnt,popcnt")
#pragma GCC optimize("Ofast")\end{lstlisting}

\subsection{fanzhuan}
\thispagestyle{fancy}

\noindent\begin{lstlisting}
auto f = [](unsigned int x) {
    x = ((x & 0xaaaaaaaa) >> 1) | ((x & 0x55555555) << 1);
    x = ((x & 0xcccccccc) >> 2) | ((x & 0x33333333) << 2);
    x = ((x & 0xf0f0f0f0) >> 4) | ((x & 0x0f0f0f0f) << 4);
    x = ((x & 0xff00ff00) >> 8) | ((x & 0x00ff00ff) << 8);
    x = ((x & 0xffff0000) >> 16) | ((x & 0x0000ffff) << 16);
    return x;
};
auto f2 = [f](unsigned int x, unsigned int k) {
    return f(x << (32 - k)) | (x >> k << k);
};
auto f=[](unsigned  long long x){
    x = ((x & 0xaaaaaaaa) >> 1) | ((x & 0x55555555) << 1);
    x = ((x & 0xcccccccc) >> 2) | ((x & 0x33333333) << 2);
    x = ((x & 0xf0f0f0f0) >> 4) | ((x & 0x0f0f0f0f) << 4);
    x = ((x & 0xff00ff00) >> 8) | ((x & 0x00ff00ff) << 8);
    x = ((x & 0xffff0000) >> 16) | ((x & 0x0000ffff) << 16);
    x=((x&0xffffffff000000))
};\end{lstlisting}

\subsection{fastIO}
\thispagestyle{fancy}

\noindent\begin{lstlisting}
namespace IO {
#define gc()(p1==p2&&(p2=(p1=buf)+fread(buf,1,1<<21,stdin),p1==p2)?EOF:*p1++)
char buf[1 << 21], *p1 = buf, *p2 = buf;
struct ifast {
    template<class T>
    ifast& operator>>(T &re) {
        re = 0;
        char ch = gc();
        while (!isdigit(ch)) ch = gc();
        while (isdigit(ch)) re = re * 10 + ch - '0', ch = gc();
        return *this;
    }
};
ifast cin;
#define pc(x)(*(p2=buf)+fwrite(buf,1,p2-buf,stdout))
struct ofast {
    template<class T>
    ofast& operator<<(T x) {
        if (x == 0) {
            pc('0');
            return *this;
        }
        static int st[60], top;
        top = 0;
        while (x) st[++top] = x % 10, x /= 10;
        for (int i = top; i >= 1; i--) pc('0' + st[i]);
        return *this;
    }

    ofast& operator<<(char div) {
        pc(div);
        return *this;
    }
};
ofast cout;
}

using namespace IO;
#define cin IO::cin
#define cout IO::cout\end{lstlisting}

\subsection{head}
\thispagestyle{fancy}

\noindent\begin{lstlisting}
#include<bits/extc++.h>
using namespace __gnu_pbds;
using namespace std;
using ll = long long;
#define LNF 0x3f3f3f3f3f3f3f3f
void chmax(auto & a, const auto &... b) {((a = (b > a ? b : a)), ...);}
void chmin(auto & a, const auto &... b) {((a = (b < a ? b : a)), ...);}
#define pb push_back
#define dbg(x) cerr<<#x<<'='<<x
void solve(void) {

}
int main() {
    ios::sync_with_stdio(false); cin.tie(nullptr);
    int t = 1;
    // cin>>t;
    while (t--)
        solve();
    return 0;
}\end{lstlisting}

\subsection{取整}
\thispagestyle{fancy}

\noindent\begin{lstlisting}
auto ceils(auto n, auto m) {
    if (m < 0) {
        n *= -1;
        m *= -1;
    }
    if (n <= 0)return n / m;
    else return (n + m - 1) / m;
}
auto floors(auto n, auto m) {
    if (m < 0) {
        n *= -1;
        m *= -1;
    }
    if (n <= 0)return (n - m + 1) / m;
    else return n / m;
}\end{lstlisting}

\subsection{离散化}
\thispagestyle{fancy}

\noindent\begin{lstlisting}
template <typename T> struct Compress {
    int n;
    vector<T> alls;
    Compress() {}
    void add(auto ...x) {
        ((alls.emplace_back(x)), ...);
    }
    void init() {
        add(numeric_limits<T>::min());
        add(numeric_limits<T>::max());
        sort(alls.begin(), alls.end());
        alls.erase(unique(alls.begin(), alls.end()), alls.end());
        this->n = alls.size() - 2;
    }
    int operator[](T x) {
        return upper_bound(alls.begin(), alls.end(), x) - alls.begin() - 1;
    }
    T operator()(int x) {
        return alls[x];
    }
};\end{lstlisting}

\section{图论}
\thispagestyle{fancy}

\subsection{Boruvka}
\thispagestyle{fancy}

\noindent\begin{lstlisting}
template <typename T>
struct Boruvka {
    int n;
    DSU c;
    const T INF;
    explicit Boruvka(size_t n, T INF = numeric_limits<T>::max())
        : n(n), c(n + 1), INF(INF) {}

    inline int find(int k) { return c.find(k); }

    template <typename F>
    T build(const F &update) {
        T cost = T{};
        while (c.size(1) < n) {
            vector<pair<T, int> > v(n + 1, make_pair(INF, -1));
            update(n);
            bool con = false;
            for (int i = 1; i <= n; i++) {
                if (v[i].second >= 1 && c.merge(i, v[i].second)) {
                    cost += v[i].first;
                    con = true;
                }
            }
            if (!con) return INF;
        }
        return cost;
    }
};\end{lstlisting}

\subsection{O1 LCA}
\thispagestyle{fancy}

\noindent\begin{lstlisting}
struct Trh {
    std::vector<int> dep, parent, in;
    int cur, n;
    int logn;
    std::vector<std::vector<int>> e;
    vector<vector<int>>a;
    Trh(int _n) : n(_n),dep(_n), parent(_n, -1), e(_n), in(_n), cur(1) {
        logn = std::__lg(n);
        a.assign(logn + 1, std::vector<int>(n + 1));
    }
    void addEdge(int u, int v) {
        e[u].push_back(v);
        e[v].push_back(u);
    }
    void dfs (int x) {
        in[x] =cur++;
        if (cur > 1) {
            a[0][cur - 2] = parent[x];
        }
        for (auto y : e[x]) {
            if (y == parent[x]) {
                continue;
            }
            parent[y] = x;
            dep[y] = dep[x] + 1;
            dfs(y);
        }
    }
    void init(int s) {
        dfs(s);
        for (int j = 0; j < logn; j++) {
            for (int i = 1; i + (2 << j) <= n; i++) {
                a[j + 1][i] = dep[a[j][i]] < dep[a[j][i + (1 << j)]] ? a[j][i] : a[j][i + (1 << j)];
            }
        }
    }
    int  lca(int x, int y) {
        if (x == y) {
            return x;
        }
        if (in[x] > in[y]) {
            std::swap(x, y);
        }
        int k = std::__lg(in[y] - in[x]);
        int u = a[k][in[x]];
        int v = a[k][in[y] - (1 << k)];
        return dep[u] < dep[v] ? u : v;
    }
};\end{lstlisting}

\subsection{On-O1 lca}
\thispagestyle{fancy}

\noindent\begin{lstlisting}
constexpr int maxn=1e5+7;
int dval[maxn], dfn[maxn], tot,Dfn[maxn];
namespace Rmq {
    int st[20][maxn / 32];
    int pre[maxn], p[maxn], w[maxn];
    inline int getmin(int x, int y) { return dfn[x] < dfn[y] ? x : y; }
    inline void down(int & x, int y) { if (dfn[x] > dfn[y]) x = y; }
    inline int qry(int l, int r) {
        const int lg = std::__lg(r - l);
        return l >= r ? 0 : getmin(st[lg][l], st[lg][r - (1 << lg)]);
    }
    inline int rmq(int l, int r) {
        if (l >> 5 == r >> 5) return p[l + __builtin_ctz(w[r] >> l)];
        else return getmin(qry((l >> 5) + 1, r >> 5), getmin(dval[l], pre[r]));
    }
    inline void build(int n) {
        ++ (n |= 31);
        memcpy(p, dval, n << 2);
        for (int i = 0; i < n; i += 32) {
            static int st[33];
            pre[i] = dval[i];
            int * top = st + 1, s = 1; w[*top = i] = s;
            for (int j = i + 1; j < i + 32; ++j) {
                for (; top != st && dfn[dval[j]] < dfn[dval[*top]]; --top) s ^= 1 << *top;
                w[j] = s |= 1 << j; *++top = j; pre[j] = dval[st[1]];
            }
            for (int j = i + 30; j >= i; --j) down(dval[j], dval[j + 1]);
            Rmq::st[0][i >> 5] = dval[i];
        }
        for (int i = 1; i < 15; ++i)
            for (int j = 0; j + (1 << i) - 1 <= n / 32; ++j)
                st[i][j] = getmin(st[i - 1][j], st[i - 1][j + (1 << i - 1)]);
    }
}
struct T { int to, nxt; } way[maxn << 1];
int h[maxn], num;
inline void addEdge(int x, int y) {
    way[++num] = {y, h[x]}, h[x] = num;
    way[++num] = {x, h[y]}, h[y] = num;
}
inline void dfs(int x, int f) {
    dval[tot] = f; dfn[x] = ++tot;
    for (int i = h[x]; i; i = way[i].nxt) if (way[i].to != f)
            dfs(way[i].to, x);
    Dfn[x]=tot;
}
inline int lca(int x, int y) {
    if (dfn[x] > dfn[y]) std::swap(x, y);
    return x == y ? x : Rmq::rmq(dfn[x], dfn[y] - 1);
}
void init() {
    for (int i = 0; i <= num; i++) {
        way[i].to = 0;
        way[i].nxt = 0;
    }
    for (int i = 0; i <= tot; i++) {
        dval[i] = 0;
        h[i] = 0;
    }
    tot = 0;
    num = 0;
}
void Getfa(int s,int n) {
    dfs(s, 0); *dfn = 1e9; Rmq::build(n - 1);
}\end{lstlisting}

\subsection{jiangly费用流}
\thispagestyle{fancy}

\noindent\begin{lstlisting}
template<class T>
struct MinCostFlow {
    struct _Edge {
        int to;
        ll cap;
        T cost;
        _Edge(int to_, ll cap_, T cost_) : to(to_), cap(cap_), cost(cost_) {}
    };
    int n;
    std::vector<_Edge> e;
    std::vector<std::vector<int>> g;
    std::vector<T> h, dis;
    std::vector<int> pre;
    bool dijkstra(int s, int t) {
        dis.assign(n, std::numeric_limits<T>::max());
        pre.assign(n, -1);
        std::priority_queue<std::pair<T, int>, std::vector<std::pair<T, int>>, std::greater<std::pair<T, int>>> que;
        dis[s] = 0;
        que.emplace(0, s);
        while (!que.empty()) {
            T d = que.top().first;
            int u = que.top().second;
            que.pop();
            if (dis[u] != d) {
                continue;
            }
            for (int i : g[u]) {
                int v = e[i].to;
                ll cap = e[i].cap;
                T cost = e[i].cost;
                if (cap > 0 && dis[v] > d + h[u] - h[v] + cost) {
                    dis[v] = d + h[u] - h[v] + cost;
                    pre[v] = i;
                    que.emplace(dis[v], v);
                }
            }
        }
        return dis[t] != std::numeric_limits<T>::max();
    }
    MinCostFlow() {}
    MinCostFlow(int n_) {
        init(n_);
    }
    void init(int n_) {
        n = n_;
        e.clear();
        g.assign(n, {});
    }
    void addEdge(int u, int v, ll cap, T cost) {
        g[u].push_back(e.size());
        e.emplace_back(v, cap, cost);
        g[v].push_back(e.size());
        e.emplace_back(u, 0, -cost);
    }
    std::pair<ll, T> flow(int s, int t) {
        ll flow = 0;
        T cost = 0;
        h.assign(n, 0);
        while (dijkstra(s, t)) {
            for (int i = 0; i < n; ++i) {
                h[i] += dis[i];
            }
            ll aug = std::numeric_limits<ll>::max();
            for (int i = t; i != s; i = e[pre[i] ^ 1].to) {
                aug = std::min(aug, e[pre[i]].cap);
            }
            for (int i = t; i != s; i = e[pre[i] ^ 1].to) {
                e[pre[i]].cap -= aug;
                e[pre[i] ^ 1].cap += aug;
            }
            flow += aug;
            cost += aug * h[t];
        }
        return std::make_pair(flow, cost);
    }
    struct Edge {
        int from;
        int to;
        T cap;
        T cost;
        T flow;
    };
    std::vector<Edge> edges() {
        std::vector<Edge> a;
        for (int i = 0; i < e.size(); i += 2) {
            Edge x;
            x.from = e[i + 1].to;
            x.to = e[i].to;
            x.cap = e[i].cap + e[i + 1].cap;
            x.cost = e[i].cost;
            x.flow = e[i + 1].cap;
            a.push_back(x);
        }
        return a;
    }
};\end{lstlisting}

\subsection{tarjan求scc}
\thispagestyle{fancy}

\noindent\begin{lstlisting}
struct SCC {
    int n;
    std::vector<std::vector<int>> adj;
    std::vector<int> stk;
    std::vector<int> dfn, low, bel;
    int cur, cnt;

    SCC() {}
    SCC(int n) {
        init(n);
    }

    void init(int n) {
        this->n = n;
        adj.assign(n, {});
        dfn.assign(n, -1);
        low.resize(n);
        bel.assign(n, -1);
        stk.clear();
        cur = cnt = 0;
    }

    void addEdge(int u, int v) {
        adj[u].push_back(v);
    }

    void dfs(int x) {
        dfn[x] = low[x] = cur++;
        stk.push_back(x);

        for (auto y : adj[x]) {
            if (dfn[y] == -1) {
                dfs(y);
                low[x] = std::min(low[x], low[y]);
            } else if (bel[y] == -1) {
                low[x] = std::min(low[x], dfn[y]);
            }
        }

        if (dfn[x] == low[x]) {
            int y;
            do {
                y = stk.back();
                bel[y] = cnt;
                stk.pop_back();
            } while (y != x);
            cnt++;
        }
    }

    std::vector<int> work() {
        for (int i = 0; i < n; i++) {
            if (dfn[i] == -1) {
                dfs(i);
            }
        }
        return bel;
    }
};\end{lstlisting}

\subsection{tarjan求点双}
\thispagestyle{fancy}

\noindent\begin{lstlisting}
constexpr int N = 1e5 + 6;
std::vector<int> e[N];
int dfn[N], low[N], idx,cut[N],sz;
void dfs(int u, int f) {
	dfn[u] = low[u] = ++idx;
	int ch = 0;
	for (auto  v : e[u]) {
		if (!dfn[v]){
			dfs(v,u);
			ch++;
			low[u]=min(low[u],low[v]);
			if(low[v]>=dfn[u])cut[u]=1;
		}else if(v!=f){
			low[u]=min(low[u],dfn[v]);
		}
	}
	if(u==1 and ch<=1)cut[u]=0;
	sz+=cut[u];
}
\end{lstlisting}

\subsection{tarjan求边双}
\thispagestyle{fancy}

\noindent\begin{lstlisting}
struct EBCC {
    int n;
    std::vector<std::vector<int>> adj;
    std::vector<int> stk;
    std::vector<int> dfn, low, bel;
    int cur, cnt;

    EBCC() {}
    EBCC(int n) {
        init(n);
    }

    void init(int n) {
        this->n = n;
        adj.assign(n, {});
        dfn.assign(n, -1);
        low.resize(n);
        bel.assign(n, -1);
        stk.clear();
        cur = cnt = 0;
    }

    void addEdge(int u, int v) {
        adj[u].push_back(v);
        adj[v].push_back(u);
    }

    void dfs(int x, int p) {
        dfn[x] = low[x] = cur++;
        stk.push_back(x);
        for (auto y : adj[x]) {
            if (y == p) {
                continue;
            }
            if (dfn[y] == -1) {
                dfs(y, x);
                low[x] = std::min(low[x], low[y]);
            } else if (bel[y] == -1 && dfn[y] < dfn[x]) {
                low[x] = std::min(low[x], dfn[y]);
            }
        }

        if (dfn[x] == low[x]) {
            int y;
            do {
                y = stk.back();
                bel[y] = cnt;
                stk.pop_back();
            } while (y != x);
            cnt++;
        }
    }

    std::vector<int> work() {
        for (int i = 1; i < n; i++) {
            if (dfn[i] == -1) {
                dfs(i, 0);
            }
        }
        return bel;
    }

    struct Graph {
        int n;
        std::vector<std::pair<int, int>> edges;
        std::vector<int> siz;
        std::vector<int> cnte;
    };
    Graph compress() {
        Graph g;
        g.n = cnt;
        g.siz.resize(cnt);
        g.cnte.resize(cnt);
        for (int i = 1; i < n; i++) {
            g.siz[bel[i]]++;
            for (auto j : adj[i]) {
                if (bel[i] < bel[j]) {
                    g.edges.emplace_back(bel[i], bel[j]);
                } else if (i < j) {
                    g.cnte[bel[i]]++;
                }
            }
        }
        return g;
    }
};\end{lstlisting}

\subsection{twosat}
\thispagestyle{fancy}

\noindent\begin{lstlisting}
struct TwoSat {
    int n;
    std::vector<std::vector<int>> e;
    std::vector<bool> ans;
    TwoSat(int n) : n(n), e(2 * n), ans(n) {}
    void addClause(int u, bool f, int v, bool g) {
        e[2 * u + !f].push_back(2 * v + g);
        e[2 * v + !g].push_back(2 * u + f);
    }
    bool satisfiable() {
        std::vector<int> id(2 * n, -1), dfn(2 * n, -1), low(2 * n, -1);
        std::vector<int> stk;
        int now = 0, cnt = 0;
        std::function<void(int)> tarjan = [&](int u) {
            stk.push_back(u);
            dfn[u] = low[u] = now++;
            for (auto v : e[u]) {
                if (dfn[v] == -1) {
                    tarjan(v);
                    low[u] = std::min(low[u], low[v]);
                } else if (id[v] == -1) {
                    low[u] = std::min(low[u], dfn[v]);
                }
            }
            if (dfn[u] == low[u]) {
                int v;
                do {
                    v = stk.back();
                    stk.pop_back();
                    id[v] = cnt;
                } while (v != u);
                ++cnt;
            }
        };
        for (int i = 0; i < 2 * n; ++i) if (dfn[i] == -1) tarjan(i);
        for (int i = 0; i < n; ++i) {
            if (id[2 * i] == id[2 * i + 1]) return false;
            ans[i] = id[2 * i] > id[2 * i + 1];
        }
        return true;
    }
    std::vector<bool> answer() { return ans; }
};
 \end{lstlisting}

\subsection{倍增lca}
\thispagestyle{fancy}

\noindent\begin{lstlisting}
template<typename T> struct Tre
{
    int n, m = 0;
    vector<vector<int>> e;
    vector<int> to;
    vector<T> wt;
    int maxk;
    vector<vector<int>> fa;
    vector<int> dep;
    void add_arc(const int u, const int v, const T w = 0) {e[u].push_back(m++); to.push_back(v); wt.push_back(w);}
    void add_edge(const int u, const int v, const T w = 0) {add_arc(u, v, w); add_arc(v, u, w);}
    Tre(const int n, const int maxk = 25): n(n), e(n), maxk(maxk), fa(n, vector<int>(maxk + 1, -1)), dep(n) {}
    void dfs(const int u, const int f)
    {
        fa[u][0] = f;
        dep[u] = f == -1 ? 0 : dep[f] + 1;
        for (int i = 1; i <= maxk; i++) fa[u][i] = fa[u][i - 1] == -1 ? -1 : fa[fa[u][i - 1]][i - 1];
        for (const int i : e[u])
        {
            const int v = to[i];
            if (v == f) continue;
            dfs(v, u);
        }
    }

    int lca(int u, int v) const
    {
        if (dep[u] < dep[v]) swap(u, v);
        for (int i = maxk; i >= 0; i--)
        {
            if (fa[u][i] != -1 && dep[fa[u][i]] >= dep[v]) u = fa[u][i];
        }
        if (u == v) return u;
        for (int i = maxk; i >= 0; i--)
        {
            if (fa[u][i] != fa[v][i]) u = fa[u][i], v = fa[v][i];
        }
        return fa[u][0];
    }
};\end{lstlisting}

\subsection{并查集}
\thispagestyle{fancy}

\noindent\begin{lstlisting}
struct DSU {
    std::vector<int> fa, siz;
    DSU() {}
    DSU(int n) {
        n++;
        init(n);
    }
     
    void init(int n) {
        fa.resize(n);
        std::iota(fa.begin(), fa.end(), 0);     
        siz.assign(n, 1);
    }
     
    int find(int x) {
        while (x != fa[x]) {
            x = fa[x] = fa[fa[x]];
        }
        return x;
    }
     
    bool same(int x, int y) {
        return find(x) == find(y);
    }
     
    bool merge(int x, int y) {
        x = find(x);
        y = find(y);
        if (x == y) {
            return false;
        }
        siz[x] += siz[y];
        fa[y] = x;
        return true;
    }
     
    int size(int x) {
        return siz[find(x)];
    }
};\end{lstlisting}

\subsection{最大流}
\thispagestyle{fancy}

\noindent\begin{lstlisting}
template<class T>
struct Flow {
    const int n;
    struct Edge {
        int to;
        T cap;
        Edge(int to, T cap) : to(to), cap(cap) {}
    };
    std::vector<Edge> e;
    std::vector<std::vector<int>> g;
    std::vector<int> cur, h;
    Flow(int n) : n(n), g(n) {}
    bool bfs(int s, int t) {
        h.assign(n, -1);
        std::queue<int> que;
        h[s] = 0;
        que.push(s);
        while (!que.empty()) {
            const int u = que.front();
            que.pop();
            for (int i : g[u]) {
                auto [v, c] = e[i];
                if (c > 0 && h[v] == -1) {
                    h[v] = h[u] + 1;
                    if (v == t) {
                        return true;
                    }
                    que.push(v);
                }
            }
        }
        return false;
    }

    T dfs(int u, int t, T f) {
        if (u == t) {
            return f;
        }
        auto r = f;
        for (int &i = cur[u]; i < int(g[u].size()); ++i) {
            const int j = g[u][i];
            auto [v, c] = e[j];
            if (c > 0 && h[v] == h[u] + 1) {
                auto a = dfs(v, t, std::min(r, c));
                e[j].cap -= a;
                e[j ^ 1].cap += a;
                r -= a;
                if (r == 0) {
                    return f;
                }
            }
        }
        return f - r;
    }
    void addEdge(int u, int v, T c) {
        g[u].push_back(e.size());
        e.emplace_back(v, c);
        g[v].push_back(e.size());
        e.emplace_back(u, 0);
    }
    T maxFlow(int s, int t) {
        T ans = 0;
        while (bfs(s, t)) {
            cur.assign(n, 0);
            ans += dfs(s, t, std::numeric_limits<T>::max());
        }
        return ans;
    }
};\end{lstlisting}

\section{字符串}
\thispagestyle{fancy}

\subsection{PAM}
\thispagestyle{fancy}

\noindent\begin{lstlisting}
struct PAM {
    static constexpr int ALPHABET_SIZE = 26;
    struct Node {
        int len;
        int link;
        int cnt;
        std::array<int, ALPHABET_SIZE> next;
        Node() : len{}, link{}, cnt{}, next{} {}
    };
    std::vector<Node> t;
    int suff;
    std::string s;
    PAM() {
        init();
    }
    void init() {
        t.assign(2, Node());
        t[0].len = -1;
        suff = 1;
        s.clear();
    }
    int newNode() {
        t.emplace_back();
        return t.size() - 1;
    }
    bool add(char c) {
        int pos = s.size();
        s += c;
        int let = c - 'a';
        int cur = suff, curlen = 0;
        while (true) {
            curlen = t[cur].len;
            if (pos - 1 - curlen >= 0 && s[pos - 1 - curlen] == s[pos]) {
                break;
            }
            cur = t[cur].link;
        }
        if (t[cur].next[let]) {
            suff = t[cur].next[let];
            return false;
        }
        int num = newNode();
        suff = num;
        t[num].len = t[cur].len + 2;
        t[cur].next[let] = num;
        if (t[num].len == 1) {
            t[num].link = 1;
            t[num].cnt = 1;
            return true;
        }
        while (true) {
            cur = t[cur].link;
            curlen = t[cur].len;
            if (pos - 1 - curlen >= 0 && s[pos - 1 - curlen] == s[pos]) {
                t[num].link = t[cur].next[let];
                break;
            }
        }
        t[num].cnt = 1 + t[t[num].link].cnt;
        return true;
    }
    int next(int p, int x) {
        return t[p].next[x];
    }
    int link(int p) {
        return t[p].link;
    }
    int len(int p) {
        return t[p].len;
    }
    int size() {
        return t.size();
    }
};\end{lstlisting}

\subsection{acm}
\thispagestyle{fancy}

\noindent\begin{lstlisting}
struct AhoCorasick {
    static constexpr int ALPHABET = 26;
    struct Node {
        int len;
        int link;
        std::array<int, ALPHABET> next;
        Node() : len{0}, link{0}, next{} {}
    };
    
    std::vector<Node> t;
    
    AhoCorasick() {
        init();
    }
    
    void init() {
        t.assign(2, Node());
        t[0].next.fill(1);
        t[0].len = -1;
    }
    
    int newNode() {
        t.emplace_back();
        return t.size() - 1;
    }
    
    int add(const std::string &a) {
        int p = 1;
        for (auto c : a) {
            int x = c - 'a';
            if (t[p].next[x] == 0) {
                t[p].next[x] = newNode();
                t[t[p].next[x]].len = t[p].len + 1;
            }
            p = t[p].next[x];
        }
        return p;
    }
    
    void work() {
        std::queue<int> q;
        q.push(1);
        
        while (!q.empty()) {
            int x = q.front();
            q.pop();
            
            for (int i = 0; i < ALPHABET; i++) {
                if (t[x].next[i] == 0) {
                    t[x].next[i] = t[t[x].link].next[i];
                } else {
                    t[t[x].next[i]].link = t[t[x].link].next[i];
                    q.push(t[x].next[i]);
                }
            }
        }
    }
    
    int next(int p, int x) {
        return t[p].next[x];
    }
    
    int link(int p) {
        return t[p].link;
    }
    
    int len(int p) {
        return t[p].len;
    }
    
    int size() {
        return t.size();
    }
};\end{lstlisting}

\subsection{hashhuiwen}
\thispagestyle{fancy}

\noindent\begin{lstlisting}
struct HHash{
    Hash h1,h2;
    int n;
    HHash(string s): h1(s), h2(string(s.rbegin(), s.rend())), n(s.length()){}
    bool check(int i, int j) {
        if (h1.getHash(i, j) == h2.getHash(n - j + 1, n - i + 1)) {
            return true;
        }
        return false;
    }
};\end{lstlisting}

\subsection{kmp}
\thispagestyle{fancy}

\noindent\begin{lstlisting}
std::vector<int> kmp(std::string s) {
    int n = s.size();
    std::vector<int> f(n + 1);
    for (int i = 1, j = 0; i < n; i++) {
        while (j && s[i] != s[j]) {
            j = f[j];
        }
        j += (s[i] == s[j]);
        f[i + 1] = j;
    }
    return f;
}\end{lstlisting}

\subsection{kmp自动机}
\thispagestyle{fancy}

\noindent\begin{lstlisting}
auto autokmp(string s) {
    ll m=s.size();
    std::vector<array<int, 26>> nxt(m + 1);
    vector<int>fail(m + 1);
    for (int i = 1; i <= m; i++) {
        int x = s[i] - 'a';
        for (int j = 0; j < 26; j++) {
            if (x == j) {
                fail[i] = nxt[fail[i - 1]][x];
                nxt[i - 1][x] = i;
            } else {
                nxt[i - 1][j] = nxt[fail[i - 1]][j];
            }
        }
    }
    return nxt;
}\end{lstlisting}

\subsection{mancher}
\thispagestyle{fancy}

\noindent\begin{lstlisting}
std::vector<int> manacher(std::string s) {
    std::string t = "#";
    for (auto c : s) {
        t += c;
        t += '#';
    }
    int n = t.size();
    std::vector<int> r(n);
    for (int i = 0, j = 0; i < n; i++) {
        if (2 * j - i >= 0 && j + r[j] > i) {
            r[i] = std::min(r[2 * j - i], j + r[j] - i);
        }
        while (i - r[i] >= 0 && i + r[i] < n && t[i - r[i]] == t[i + r[i]]) {
            r[i] += 1;
        }
        if (i + r[i] > j + r[j]) {
            j = i;
        }
    }
    return r;
}
\end{lstlisting}

\subsection{sa+rmq求lcp}
\thispagestyle{fancy}

\noindent\begin{lstlisting}
struct SuffixArray {
    int n;
    std::vector<int> sa, rk, lc;
    SuffixArray(const std::string &s) {
        n = s.length();
        sa.resize(n);
        lc.resize(n - 1);
        rk.resize(n);
        std::iota(sa.begin(), sa.end(), 0);
        std::sort(sa.begin(), sa.end(),
                  [&](int a, int b) {
                      return s[a] < s[b];
                  });
        rk[sa[0]] = 0;
        for (int i = 1; i < n; i++) {
            rk[sa[i]] = rk[sa[i - 1]] + (s[sa[i]] != s[sa[i - 1]]);
        }
        int k = 1;
        std::vector<int> tmp, cnt(n);
        tmp.reserve(n);
        while (rk[sa[n - 1]] < n - 1) {
            tmp.clear();
            for (int i = 0; i < k; i++) {
                tmp.push_back(n - k + i);
            }
            for (auto i : sa) {
                if (i >= k) {
                    tmp.push_back(i - k);
                }
            }
            std::fill(cnt.begin(), cnt.end(), 0);
            for (int i = 0; i < n; i++) {
                cnt[rk[i]]++;
            }
            for (int i = 1; i < n; i++) {
                cnt[i] += cnt[i - 1];
            }
            for (int i = n - 1; i >= 0; i--) {
                sa[--cnt[rk[tmp[i]]]] = tmp[i];
            }
            std::swap(rk, tmp);
            rk[sa[0]] = 0;
            for (int i = 1; i < n; i++) {
                rk[sa[i]] = rk[sa[i - 1]] + (tmp[sa[i - 1]] < tmp[sa[i]] || sa[i - 1] + k == n || tmp[sa[i - 1] + k] < tmp[sa[i] + k]);
            }
            k *= 2;
        }
        for (int i = 0, j = 0; i < n; i++) {
            if (rk[i] == 0) {
                j = 0;
            } else {
                for (j -= (j > 0); i + j < n && sa[rk[i] - 1] + j < n && s[i + j] == s[sa[rk[i] - 1] + j]; j++)
                    ;
                lc[rk[i] - 1] = j;
            }
        }
    }
};

template<class T,
        class Cmp = std::less<T>>
struct RMQ {
    const Cmp cmp = Cmp();
    static constexpr unsigned B = 64;
    using u64 = unsigned long long;
    int n;
    std::vector<std::vector<T>> a;
    std::vector<T> pre, suf, ini;
    std::vector<u64> stk;
    RMQ() {}
    RMQ(const std::vector<T> &v) {
        init(v);
    }
    void init(const std::vector<T> &v) {
        n = v.size();
        pre = suf = ini = v;
        stk.resize(n);
        if (!n) {
            return;
        }
        const int M = (n - 1) / B + 1;
        const int lg = std::__lg(M);
        a.assign(lg + 1, std::vector<T>(M));
        for (int i = 0; i < M; i++) {
            a[0][i] = v[i * B];
            for (int j = 1; j < B && i * B + j < n; j++) {
                a[0][i] = std::min(a[0][i], v[i * B + j], cmp);
            }
        }
        for (int i = 1; i < n; i++) {
            if (i % B) {
                pre[i] = std::min(pre[i], pre[i - 1], cmp);
            }
        }
        for (int i = n - 2; i >= 0; i--) {
            if (i % B != B - 1) {
                suf[i] = std::min(suf[i], suf[i + 1], cmp);
            }
        }
        for (int j = 0; j < lg; j++) {
            for (int i = 0; i + (2 << j) <= M; i++) {
                a[j + 1][i] = std::min(a[j][i], a[j][i + (1 << j)], cmp);
            }
        }
        for (int i = 0; i < M; i++) {
            const int l = i * B;
            const int r = std::min(1U * n, l + B);
            u64 s = 0;
            for (int j = l; j < r; j++) {
                while (s && cmp(v[j], v[std::__lg(s) + l])) {
                    s ^= 1ULL << std::__lg(s);
                }
                s |= 1ULL << (j - l);
                stk[j] = s;
            }
        }
    }
    T operator()(int l, int r) {
        if (l / B != (r - 1) / B) {
            T ans = std::min(suf[l], pre[r - 1], cmp);
            l = l / B + 1;
            r = r / B;
            if (l < r) {
                int k = std::__lg(r - l);
                ans = std::min({ans, a[k][l], a[k][r - (1 << k)]}, cmp);
            }
            return ans;
        } else {
            int x = B * (l / B);
            return ini[__builtin_ctzll(stk[r - 1] >> (l - x)) + l];
        }
    }
};
SuffixArray sa(s);

RMQ rmq(sa.lc);

auto lcp = [&](int i, int j) {
    if (i == j) {
        return n - i;
    }
    i = sa.rk[i];
    j = sa.rk[j];
    if (i > j) {
        std::swap(i, j);
    }
    return rmq(i, j);
};\end{lstlisting}

\subsection{sam}
\thispagestyle{fancy}

\noindent\begin{lstlisting}
struct SAM {
    static constexpr int ALPHABET_SIZE = 26;
    struct Node {
        int len;
        int link;
        std::array<int, ALPHABET_SIZE> next;
        Node() : len{}, link{}, next{} {}
    };
    std::vector<Node> t;
    SAM() {
        init();
    }
    void init() {
        t.assign(2, Node());
        t[0].next.fill(1);
        t[0].len = -1;
    }
    int newNode() {
        t.emplace_back();
        return t.size() - 1;
    }
    int extend(int p, int c) {
        if (t[p].next[c]) {
            int q = t[p].next[c];
            if (t[q].len == t[p].len + 1) {
                return q;
            }
            int r = newNode();
            t[r].len = t[p].len + 1;
            t[r].link = t[q].link;
            t[r].next = t[q].next;
            t[q].link = r;
            while (t[p].next[c] == q) {
                t[p].next[c] = r;
                p = t[p].link;
            }
            return r;
        }
        int cur = newNode();
        t[cur].len = t[p].len + 1;
        while (!t[p].next[c]) {
            t[p].next[c] = cur;
            p = t[p].link;
        }
        t[cur].link = extend(p, c);
        return cur;
    }
    int extend(int p, char c, char offset = 'a') {
        return extend(p, c - offset);
    }

    int next(int p, int x) {
        return t[p].next[x];
    }

    int next(int p, char c, char offset = 'a') {
        return next(p, c - 'a');
    }

    int link(int p) {
        return t[p].link;
    }

    int len(int p) {
        return t[p].len;
    }

    int size() {
        return t.size();
    }
};\end{lstlisting}

\subsection{trie}
\thispagestyle{fancy}

\noindent\begin{lstlisting}
constexpr int N = 1e6 + 7;
int tre[N][26];
int tot = 0;
int newNode() {
    tot++;
    fill(tre[tot], tre[tot] + 26, 0);
    return tot;
}
void insert(string & a) {
	int now = 1;
	for (auto q : a) {
		int &c = tre[now][q - 'a'];
		if (!c)c = newNode();
		now = c;
	}
}
\end{lstlisting}

\subsection{ull单哈}
\thispagestyle{fancy}

\noindent\begin{lstlisting}
using ull = unsigned long long;
using u128 = unsigned __int128;
constexpr ull Mod = (1ULL << 61) - 1;
constexpr int maxn = 1e6 + 7;
mt19937_64 rnd(chrono::steady_clock::now().time_since_epoch().count());
uniform_int_distribution<ull> dist(Mod / 2, Mod - 1);
const ull Seed = dist(rnd);
ull bas[maxn];
ull norm(ull x)
{
    return (x>=Mod)?x-Mod:x;
}
ull mul(ull a, ull b) {
    u128 t = (u128)(a) * b;
    t = (t >> 61) + (t & Mod);
    return (t >= Mod) ? t - Mod : t;
}
ull del(ull a,ull b){
    return norm(a-b+Mod);
}
ull add(ull a,ull b){
    return norm(a+b);
}
void HInit() {
    bas[0] = 1;
    for (int i = 1; i < maxn; i++) {
        bas[i] = mul(bas[i - 1] , Seed);
    }
}
struct Hash {
    int n;
    std::vector<ull> sum;
    Hash(const string& s): n(s.length()), sum(n + 1) {
        for (int i = 1; i <= n; i++) {
            sum[i] = add(mul(sum[i - 1] , Seed),s[i - 1]);
        }
    }
    ull getHash(int l, int r) {
        ull res = del(sum[r] - mul(sum[l - 1], bas[r - l + 1]));
        return res;
    }
};\end{lstlisting}

\subsection{后缀数组}
\thispagestyle{fancy}

\noindent\begin{lstlisting}
struct SuffixArray {
    int n;
    std::vector<int> sa, rk, lc;
    SuffixArray(const std::string &s) {
        n = s.length();
        sa.resize(n);
        lc.resize(n - 1);
        rk.resize(n);
        std::iota(sa.begin(), sa.end(), 0);
        std::sort(sa.begin(), sa.end(),
                  [&](int a, int b) {
                      return s[a] < s[b];
                  });
        rk[sa[0]] = 0;
        for (int i = 1; i < n; i++) {
            rk[sa[i]] = rk[sa[i - 1]] + (s[sa[i]] != s[sa[i - 1]]);
        }
        int k = 1;
        std::vector<int> tmp, cnt(n);
        tmp.reserve(n);
        while (rk[sa[n - 1]] < n - 1) {
            tmp.clear();
            for (int i = 0; i < k; i++) {
                tmp.push_back(n - k + i);
            }
            for (auto i : sa) {
                if (i >= k) {
                    tmp.push_back(i - k);
                }
            }
            std::fill(cnt.begin(), cnt.end(), 0);
            for (int i = 0; i < n; i++) {
                cnt[rk[i]]++;
            }
            for (int i = 1; i < n; i++) {
                cnt[i] += cnt[i - 1];
            }
            for (int i = n - 1; i >= 0; i--) {
                sa[--cnt[rk[tmp[i]]]] = tmp[i];
            }
            std::swap(rk, tmp);
            rk[sa[0]] = 0;
            for (int i = 1; i < n; i++) {
                rk[sa[i]] = rk[sa[i - 1]] + (tmp[sa[i - 1]] < tmp[sa[i]] || sa[i - 1] + k == n || tmp[sa[i - 1] + k] < tmp[sa[i] + k]);
            }
            k *= 2;
        }
        for (int i = 0, j = 0; i < n; i++) {
            if (rk[i] == 0) {
                j = 0;
            } else {
                for (j -= (j > 0); i + j < n && sa[rk[i] - 1] + j < n && s[i + j] == s[sa[rk[i] - 1] + j]; j++);
                lc[rk[i] - 1] = j;
            }
        }
    }
};\end{lstlisting}

\subsection{自动取模类hash}
\thispagestyle{fancy}

\noindent\begin{lstlisting}
using ull = unsigned long long;
using u128 = unsigned __int128;
constexpr ull Mod = (1ULL << 61) - 1;
template<typename T>
constexpr T power(T a, ull b) {
    T res {1};
    for (; b != 0; b /= 2, a *= a) {
        if (b % 2 == 1) {
            res *= a;
        }
    }
    return res;
}

ull mulMod(ull a, ull b) {
    u128 t = static_cast<u128>(a) * b;
    t = (t >> 61) + (t & Mod);
    return (t >= Mod) ? t - Mod : t;
}

template<typename U, U P>
struct ModIntBase {
public:
    constexpr ModIntBase() : x {0} {}

    template<typename T, typename = std::enable_if_t<std::is_integral<T>::value>>
    constexpr ModIntBase(T x_) : x {norm(x_ % P)} {}

    constexpr static U norm(U x) {
        if ((x >> (8 * sizeof(U) - 1) & 1) == 1) {
            x += P;
        }
        if (x >= P) {
            x -= P;
        }
        return x;
    }

    constexpr U val() const {
        return x;
    }

    constexpr ModIntBase operator-() const {
        ModIntBase res;
        res.x = norm(P - x);
        return res;
    }

    constexpr ModIntBase inv() const {
        return power(*this, P - 2);
    }

    constexpr ModIntBase &operator*=(const ModIntBase &rhs) & {
        x = mulMod(x, rhs.val());
        return *this;
    }

    constexpr ModIntBase &operator+=(const ModIntBase &rhs) & {
        x = norm(x + rhs.x);
        return *this;
    }

    constexpr ModIntBase &operator-=(const ModIntBase &rhs) & {
        x = norm(x- rhs.x+P);
        return *this;
    }

    constexpr ModIntBase &operator/=(const ModIntBase &rhs) & {
        return *this *= rhs.inv();
    }

    friend constexpr ModIntBase operator*(ModIntBase lhs, const ModIntBase &rhs) {
        lhs *= rhs;
        return lhs;
    }

    friend constexpr ModIntBase operator+(ModIntBase lhs, const ModIntBase &rhs) {
        lhs += rhs;
        return lhs;
    }

    friend constexpr ModIntBase operator-(ModIntBase lhs, const ModIntBase &rhs) {
        lhs -= rhs;
        return lhs;
    }

    friend constexpr ModIntBase operator/(ModIntBase lhs, const ModIntBase &rhs) {
        lhs /= rhs;
        return lhs;
    }

    friend constexpr std::ostream &operator<<(std::ostream &os, const ModIntBase &a) {
        return os << a.val();
    }

    friend constexpr bool operator==(ModIntBase lhs, ModIntBase rhs) {
        return lhs.val() == rhs.val();
    }

    friend constexpr bool operator!=(ModIntBase lhs, ModIntBase rhs) {
        return lhs.val() != rhs.val();
    }

    friend constexpr bool operator<(ModIntBase lhs, ModIntBase rhs) {
        return lhs.val() < rhs.val();
    }

private:
    U x;
};

template<ull P>
using ModInt64 = ModIntBase<ull, P>;
using Z = ModInt64<Mod>;
constexpr int maxn = 1e6 + 7;
mt19937_64 rnd(chrono::steady_clock::now().time_since_epoch().count());
uniform_int_distribution<ull> dist(Mod / 2, Mod - 1);
const ull Seed = dist(rnd);
Z bas[maxn];
void HInit() {
    bas[0] = Z(1);
    for (int i = 1; i < maxn; i++) {
        bas[i] = bas[i - 1] * Seed;
    }
}
struct Hash {
    int n;
    std::vector<Z> sum;
    Hash(const string& s): n(s.length()), sum(n + 1) {
        for (int i = 1; i <= n; i++) {
            sum[i] = (sum[i - 1] * Seed) + s[i - 1];
        }
    }
    Z getHash(int l, int r) {
        return (sum[r] - sum[l - 1] * bas[r - l + 1]);
    }
};\end{lstlisting}

\section{数学}
\thispagestyle{fancy}

\subsection{Mlong}
\thispagestyle{fancy}

\noindent\begin{lstlisting}
template<class T>
constexpr T power(T a, ll b) {
    T res = 1;
    for (; b; b /= 2, a *= a) {
        if (b % 2) {
            res *= a;
        }
    }
    return res;
}

constexpr ll mul(ll a, ll b, ll p) {
    ll res = a * b - ll(1.L * a * b / p) * p;
    res %= p;
    if (res < 0) {
        res += p;
    }
    return res;
}
template<ll P>
struct MLong {
    ll x;
    constexpr MLong() : x{} {}
    constexpr MLong(ll x) : x{norm(x % getMod())} {}

    static ll Mod;
    constexpr static ll getMod() {
        if (P > 0) {
            return P;
        } else {
            return Mod;
        }
    }
    constexpr static void setMod(ll Mod_) {
        Mod = Mod_;
    }
    constexpr ll norm(ll x) const {
        if (x < 0) {
            x += getMod();
        }
        if (x >= getMod()) {
            x -= getMod();
        }
        return x;
    }
    constexpr ll val() const {
        return x;
    }
    explicit constexpr operator ll() const {
        return x;
    }
    constexpr MLong operator-() const {
        MLong res;
        res.x = norm(getMod() - x);
        return res;
    }
    constexpr MLong inv() const {
        assert(x != 0);
        return power(*this, getMod() - 2);
    }
    constexpr MLong &operator*=(MLong rhs) & {
        x = mul(x, rhs.x, getMod());
        return *this;
    }
    constexpr MLong &operator+=(MLong rhs) & {
        x = norm(x + rhs.x);
        return *this;
    }
    constexpr MLong &operator-=(MLong rhs) & {
        x = norm(x - rhs.x);
        return *this;
    }
    constexpr MLong &operator/=(MLong rhs) & {
        return *this *= rhs.inv();
    }
    friend constexpr MLong operator*(MLong lhs, MLong rhs) {
        MLong res = lhs;
        res *= rhs;
        return res;
    }
    friend constexpr MLong operator+(MLong lhs, MLong rhs) {
        MLong res = lhs;
        res += rhs;
        return res;
    }
    friend constexpr MLong operator-(MLong lhs, MLong rhs) {
        MLong res = lhs;
        res -= rhs;
        return res;
    }
    friend constexpr MLong operator/(MLong lhs, MLong rhs) {
        MLong res = lhs;
        res /= rhs;
        return res;
    }
    friend constexpr std::istream &operator>>(std::istream &is, MLong &a) {
        ll v;
        is >> v;
        a = MLong(v);
        return is;
    }
    friend constexpr std::ostream &operator<<(std::ostream &os, const MLong &a) {
        return os << a.val();
    }
    friend constexpr bool operator==(MLong lhs, MLong rhs) {
        return lhs.val() == rhs.val();
    }
    friend constexpr bool operator!=(MLong lhs, MLong rhs) {
        return lhs.val() != rhs.val();
    }
};

template<>
ll MLong<0LL>::Mod = ll(1E18) + 9;

template<int V, int P>
constexpr MLong<P> CInv = MLong<P>(V).inv();

constexpr ll P = ll(1E18) + 9;
using Z = MLong<P>;
 \end{lstlisting}

\subsection{Pollard-pho}
\thispagestyle{fancy}

\noindent\begin{lstlisting}
ll mul(ll a, ll b, ll m) {
    return static_cast<__int128>(a) * b % m;
}
ll power(ll a, ll b, ll m) {
    ll res = 1 % m;
    for (; b; b >>= 1, a = mul(a, a, m))
        if (b & 1)
            res = mul(res, a, m);
    return res;
}
bool isprime(ll n) {
    if (n < 2)
        return false;
    static constexpr int A[] = {2, 3, 5, 7, 11, 13, 17, 19, 23};
    int s = __builtin_ctzll(n - 1);
    ll d = (n - 1) >> s;
    for (auto a : A) {
        if (a == n)
            return true;
        ll x = power(a, d, n);
        if (x == 1 || x == n - 1)
            continue;
        bool ok = false;
        for (int i = 0; i < s - 1; ++i) {
            x = mul(x, x, n);
            if (x == n - 1) {
                ok = true;
                break;
            }
        }
        if (!ok)
            return false;
    }
    return true;
}
std::vector<ll> factorize(ll n) {
    std::vector<ll> p;
    std::function<void(ll)> f = [&](ll n) {
        if (n <= 10000) {
            for (int i = 2; i * i <= n; ++i)
                for (; n % i == 0; n /= i)
                    p.push_back(i);
            if (n > 1)
                p.push_back(n);
            return;
        }
        if (isprime(n)) {
            p.push_back(n);
            return;
        }
        auto g = [&](ll x) {
            return (mul(x, x, n) + 1) % n;
        };
        ll x0 = 2;
        while (true) {
            ll x = x0;
            ll y = x0;
            ll d = 1;
            ll power = 1, lam = 0;
            ll v = 1;
            while (d == 1) {
                y = g(y);
                ++lam;
                v = mul(v, std::abs(x - y), n);
                if (lam % 127 == 0) {
                    d = std::gcd(v, n);
                    v = 1;
                }
                if (power == lam) {
                    x = y;
                    power *= 2;
                    lam = 0;
                    d = std::gcd(v, n);
                    v = 1;
                }
            }
            if (d != n) {
                f(d);
                f(n / d);
                return;
            }
            ++x0;
        }
    };
    f(n);
    std::sort(p.begin(), p.end());
    vector<ll>pp=p;
    vector<ll>ppp;
    pp.erase(unique(pp.begin(),pp.end()),pp.end());
    for(auto q:pp){
        ll w=count(p.begin(),p.end(),q);
        ll res=1;
        while(w){
            res*=q;
            w--;
        }
        ppp.push_back(res);
    }
    return p;
}\end{lstlisting}

\subsection{ax+by=c求解}
\thispagestyle{fancy}

\noindent\begin{lstlisting}
pair<ll,ll> exgcd(ll a, ll b, ll c)
{
    function<ll(ll, ll, ll &, ll &)> ex_gcd = [&](ll a, ll b, ll & x, ll & y)
    {
        if (b == 0)
        {
            x = 1;
            y = 0;
            return a;
        }
        ll x1, y1;
        ll g = ex_gcd(b, a % b, x1, y1);
        x = y1;
        y = x1 - a / b * y1;
        return g;
    };
    ll x, y;
    ll g = ex_gcd(a, b, x, y);
    if (c % g != 0)
        return {0, 0};
    ll z = abs(b / g);
    x = (__int128_t)x * (c / g) % z;
    x = (x + z) % z;
    ll w = abs(a / g);
    y = (__int128_t)y * (c / g) % w;
    y = (y + w) % w;
    ll xmin = x, ymin = y;
    ll xmax = (c - (__int128_t)b * ymin) / a;
    ll ymax = (c - (__int128_t)a * xmin) / b;
    return {x, z};
}\end{lstlisting}

\subsection{jly多项式}
\thispagestyle{fancy}

\noindent\begin{lstlisting}
#include <bits/stdc++.h>

constexpr int P = 998244353;
using i64 = long long;
// assume -P <= x < 2P
int norm(int x) {
    if (x < 0) {
        x += P;
    }
    if (x >= P) {
        x -= P;
    }
    return x;
}
template<class T>
T power(T a, int b) {
    T res = 1;
    for (; b; b /= 2, a *= a) {
        if (b % 2) {
            res *= a;
        }
    }
    return res;
}
struct Z {
    int x;
    Z(int x = 0) : x(norm(x)) {}
    int val() const {
        return x;
    }
    Z operator-() const {
        return Z(norm(P - x));
    }
    Z inv() const {
        assert(x != 0);
        return power(*this, P - 2);
    }
    Z &operator*=(const Z &rhs) {
        x = i64(x) * rhs.x % P;
        return *this;
    }
    Z &operator+=(const Z &rhs) {
        x = norm(x + rhs.x);
        return *this;
    }
    Z &operator-=(const Z &rhs) {
        x = norm(x - rhs.x);
        return *this;
    }
    Z &operator/=(const Z &rhs) {
        return *this *= rhs.inv();
    }
    friend Z operator*(const Z &lhs, const Z &rhs) {
        Z res = lhs;
        res *= rhs;
        return res;
    }
    friend Z operator+(const Z &lhs, const Z &rhs) {
        Z res = lhs;
        res += rhs;
        return res;
    }
    friend Z operator-(const Z &lhs, const Z &rhs) {
        Z res = lhs;
        res -= rhs;
        return res;
    }
    friend Z operator/(const Z &lhs, const Z &rhs) {
        Z res = lhs;
        res /= rhs;
        return res;
    }
    friend std::istream &operator>>(std::istream &is, Z &a) {
        i64 v;
        is >> v;
        a = Z(v);
        return is;
    }
    friend std::ostream &operator<<(std::ostream &os, const Z &a) {
        return os << a.val();
    }
};

std::vector<int> rev;
std::vector<Z> roots{0, 1};
void dft(std::vector<Z> &a) {
    int n = a.size();
    
    if (int(rev.size()) != n) {
        int k = __builtin_ctz(n) - 1;
        rev.resize(n);
        for (int i = 0; i < n; i++) {
            rev[i] = rev[i >> 1] >> 1 | (i & 1) << k;
        }
    }
    
    for (int i = 0; i < n; i++) {
        if (rev[i] < i) {
            std::swap(a[i], a[rev[i]]);
        }
    }
    if (int(roots.size()) < n) {
        int k = __builtin_ctz(roots.size());
        roots.resize(n);
        while ((1 << k) < n) {
            Z e = power(Z(3), (P - 1) >> (k + 1));
            for (int i = 1 << (k - 1); i < (1 << k); i++) {
                roots[2 * i] = roots[i];
                roots[2 * i + 1] = roots[i] * e;
            }
            k++;
        }
    }
    for (int k = 1; k < n; k *= 2) {
        for (int i = 0; i < n; i += 2 * k) {
            for (int j = 0; j < k; j++) {
                Z u = a[i + j];
                Z v = a[i + j + k] * roots[k + j];
                a[i + j] = u + v;
                a[i + j + k] = u - v;
            }
        }
    }
}
void idft(std::vector<Z> &a) {
    int n = a.size();
    std::reverse(a.begin() + 1, a.end());
    dft(a);
    Z inv = (1 - P) / n;
    for (int i = 0; i < n; i++) {
        a[i] *= inv;
    }
}
struct Poly {
    std::vector<Z> a;
    Poly() {}
    Poly(const std::vector<Z> &a) : a(a) {}
    Poly(const std::initializer_list<Z> &a) : a(a) {}
    int size() const {
        return a.size();
    }
    void resize(int n) {
        a.resize(n);
    }
    Z operator[](int idx) const {
        if (idx < size()) {
            return a[idx];
        } else {
            return 0;
        }
    }
    Z &operator[](int idx) {
        return a[idx];
    }
    Poly mulxk(int k) const {
        auto b = a;
        b.insert(b.begin(), k, 0);
        return Poly(b);
    }
    Poly modxk(int k) const {
        k = std::min(k, size());
        return Poly(std::vector<Z>(a.begin(), a.begin() + k));
    }
    Poly divxk(int k) const {
        if (size() <= k) {
            return Poly();
        }
        return Poly(std::vector<Z>(a.begin() + k, a.end()));
    }
    friend Poly operator+(const Poly &a, const Poly &b) {
        std::vector<Z> res(std::max(a.size(), b.size()));
        for (int i = 0; i < int(res.size()); i++) {
            res[i] = a[i] + b[i];
        }
        return Poly(res);
    }
    friend Poly operator-(const Poly &a, const Poly &b) {
        std::vector<Z> res(std::max(a.size(), b.size()));
        for (int i = 0; i < int(res.size()); i++) {
            res[i] = a[i] - b[i];
        }
        return Poly(res);
    }
    friend Poly operator*(Poly a, Poly b) {
        if (a.size() == 0 || b.size() == 0) {
            return Poly();
        }
        int sz = 1, tot = a.size() + b.size() - 1;
        while (sz < tot) {
            sz *= 2;
        }
        a.a.resize(sz);
        b.a.resize(sz);
        dft(a.a);
        dft(b.a);
        for (int i = 0; i < sz; ++i) {
            a.a[i] = a[i] * b[i];
        }
        idft(a.a);
        a.resize(tot);
        return a;
    }
    friend Poly operator*(Z a, Poly b) {
        for (int i = 0; i < int(b.size()); i++) {
            b[i] *= a;
        }
        return b;
    }
    friend Poly operator*(Poly a, Z b) {
        for (int i = 0; i < int(a.size()); i++) {
            a[i] *= b;
        }
        return a;
    }
    Poly &operator+=(Poly b) {
        return (*this) = (*this) + b;
    }
    Poly &operator-=(Poly b) {
        return (*this) = (*this) - b;
    }
    Poly &operator*=(Poly b) {
        return (*this) = (*this) * b;
    }
    Poly deriv() const {
        if (a.empty()) {
            return Poly();
        }
        std::vector<Z> res(size() - 1);
        for (int i = 0; i < size() - 1; ++i) {
            res[i] = (i + 1) * a[i + 1];
        }
        return Poly(res);
    }
    Poly integr() const {
        std::vector<Z> res(size() + 1);
        for (int i = 0; i < size(); ++i) {
            res[i + 1] = a[i] / (i + 1);
        }
        return Poly(res);
    }
    Poly inv(int m) const {
        Poly x{a[0].inv()};
        int k = 1;
        while (k < m) {
            k *= 2;
            x = (x * (Poly{2} - modxk(k) * x)).modxk(k);
        }
        return x.modxk(m);
    }
    Poly log(int m) const {
        return (deriv() * inv(m)).integr().modxk(m);
    }
    Poly exp(int m) const {
        Poly x{1};
        int k = 1;
        while (k < m) {
            k *= 2;
            x = (x * (Poly{1} - x.log(k) + modxk(k))).modxk(k);
        }
        return x.modxk(m);
    }
    Poly pow(int k, int m) const {
        int i = 0;
        while (i < size() && a[i].val() == 0) {
            i++;
        }
        if (i == size() || 1LL * i * k >= m) {
            return Poly(std::vector<Z>(m));
        }
        Z v = a[i];
        auto f = divxk(i) * v.inv();
        return (f.log(m - i * k) * k).exp(m - i * k).mulxk(i * k) * power(v, k);
    }
    Poly sqrt(int m) const {
        Poly x{1};
        int k = 1;
        while (k < m) {
            k *= 2;
            x = (x + (modxk(k) * x.inv(k)).modxk(k)) * ((P + 1) / 2);
        }
        return x.modxk(m);
    }
    Poly mulT(Poly b) const {
        if (b.size() == 0) {
            return Poly();
        }
        int n = b.size();
        std::reverse(b.a.begin(), b.a.end());
        return ((*this) * b).divxk(n - 1);
    }
    std::vector<Z> eval(std::vector<Z> x) const {
        if (size() == 0) {
            return std::vector<Z>(x.size(), 0);
        }
        const int n = std::max(int(x.size()), size());
        std::vector<Poly> q(4 * n);
        std::vector<Z> ans(x.size());
        x.resize(n);
        std::function<void(int, int, int)> build = [&](int p, int l, int r) {
            if (r - l == 1) {
                q[p] = Poly{1, -x[l]};
            } else {
                int m = (l + r) / 2;
                build(2 * p, l, m);
                build(2 * p + 1, m, r);
                q[p] = q[2 * p] * q[2 * p + 1];
            }
        };
        build(1, 0, n);
        std::function<void(int, int, int, const Poly &)> work = [&](int p, int l, int r, const Poly &num) {
            if (r - l == 1) {
                if (l < int(ans.size())) {
                    ans[l] = num[0];
                }
            } else {
                int m = (l + r) / 2;
                work(2 * p, l, m, num.mulT(q[2 * p + 1]).modxk(m - l));
                work(2 * p + 1, m, r, num.mulT(q[2 * p]).modxk(r - m));
            }
        };
        work(1, 0, n, mulT(q[1].inv(n)));
        return ans;
    }
};
\end{lstlisting}

\subsection{mint}
\thispagestyle{fancy}

\noindent\begin{lstlisting}
template<class T>
constexpr T power(T a, ll b) {
    T res = 1;
    for (; b; b /= 2, a *= a) {
        if (b % 2) {
            res *= a;
        }
    }
    return res;
}

template<int P>
struct MInt {
    int x;
    constexpr MInt() : x{} {}
    constexpr MInt(ll x) : x{norm(x % P)} {}

    constexpr int norm(int x) const {
        if (x < 0) {
            x += P;
        }
        if (x >= P) {
            x -= P;
        }
        return x;
    }
    constexpr int val() const {
        return x;
    }
    explicit constexpr operator int() const {
        return x;
    }
    constexpr MInt operator-() const {
        MInt res;
        res.x = norm(P - x);
        return res;
    }
    constexpr MInt inv() const {
        assert(x != 0);
        return power(*this, P - 2);
    }
    constexpr MInt &operator*=(MInt rhs) {
        x = 1LL * x * rhs.x % P;
        return *this;
    }
    constexpr MInt &operator+=(MInt rhs) {
        x = norm(x + rhs.x);
        return *this;
    }
    constexpr MInt &operator-=(MInt rhs) {
        x = norm(x - rhs.x);
        return *this;
    }
    constexpr MInt &operator/=(MInt rhs) {
        return *this *= rhs.inv();
    }
    friend constexpr MInt operator*(MInt lhs, MInt rhs) {
        MInt res = lhs;
        res *= rhs;
        return res;
    }
    friend constexpr MInt operator+(MInt lhs, MInt rhs) {
        MInt res = lhs;
        res += rhs;
        return res;
    }
    friend constexpr MInt operator-(MInt lhs, MInt rhs) {
        MInt res = lhs;
        res -= rhs;
        return res;
    }
    friend constexpr MInt operator/(MInt lhs, MInt rhs) {
        MInt res = lhs;
        res /= rhs;
        return res;
    }
    friend constexpr std::istream &operator>>(std::istream &is, MInt &a) {
        ll v;
        is >> v;
        a = MInt(v);
        return is;
    }
    friend constexpr std::ostream &operator<<(std::ostream &os, const MInt &a) {
        return os << a.val();
    }
    friend constexpr bool operator==(MInt lhs, MInt rhs) {
        return lhs.val() == rhs.val();
    }
    friend constexpr bool operator!=(MInt lhs, MInt rhs) {
        return lhs.val() != rhs.val();
    }
};

template<int V, int P>
constexpr MInt<P> CInv = MInt<P>(V).inv();

constexpr int P = 998244353;
// constexpr int P = 1e9+7;
using Z = MInt<P>;

struct Comb {
    int n;
    std::vector<Z> _fac;
    std::vector<Z> _invfac;
    std::vector<Z> _inv;

    Comb() : n{0}, _fac{1}, _invfac{1}, _inv{0} {}
    Comb(int n) : Comb() {
        init(n);
    }

    void init(int m) {
        if (m <= n) return;
        _fac.resize(m + 1);
        _invfac.resize(m + 1);
        _inv.resize(m + 1);

        for (int i = n + 1; i <= m; i++) {
            _fac[i] = _fac[i - 1] * i;
        }
        _invfac[m] = _fac[m].inv();
        for (int i = m; i > n; i--) {
            _invfac[i - 1] = _invfac[i] * i;
            _inv[i] = _invfac[i] * _fac[i - 1];
        }
        n = m;
    }

    Z fac(int m) {
        if (m > n) init(2 * m);
        return _fac[m];
    }
    Z invfac(int m) {
        if (m > n) init(2 * m);
        return _invfac[m];
    }
    Z inv(int m) {
        if (m > n) init(2 * m);
        return _inv[m];
    }
    Z binom(int n, int m) {
        if (n < m || m < 0) return 0;
        return fac(n) * invfac(m) * invfac(n - m);
    }
} comb;
struct Inversion {
    static constexpr int B = (1 << 10), T = (1 << 20);
    std::array < int, T + 1 > f, p;
    std::array < int, T * 3 + 3 > buf;
    int *I = buf.begin() + T;
    Inversion() {
        for (int i = 1; i <= B; i++) {
            int s = 0, d = (i << 10);
            for (int j = 1; j <= T; j++) {
                if ((s += d) >= P) s -= P;
                if (s <= T) {
                    if (!f[j]) f[j] = i, p[j] = s;
                }
                else if (s >= P - T) {
                    if (!f[j]) f[j] = i, p[j] = s - P;
                }
                else {
                    int t = (P - T - s - 1) / d;
                    s += t * d, j += t;
                }
            }
        }
        I[1] = f[0] = 1;
        for (int i = 2; i <= (T << 1); i++)
            I[i] = 1ll * (P - P / i) * I[P % i] % P;

        for (int i = -1; i >= -T; i--)
            I[i] = P - I[-i];

    }
    Z inv(int x) {
        return Z(1)* I[p[x >> 10] + (x & 1023) * f[x >> 10]] * f[x >> 10];
    }
};\end{lstlisting}

\subsection{skip,Pollard-pho}
\thispagestyle{fancy}

\noindent\begin{lstlisting}
namespace factor {
    using f64 = long double;
    ll p;
    f64 invp;
    void setmod(ll x) {
        p = x, invp = (f64) 1 / x;
    }
    ll mul(ll a, ll b) {
        ll z = a * invp * b + 0.5;
        ll res = a * b - z * p;
        return res + (res >> 63 & p);
    }
    ll power(ll a, ll x, ll res = 1) {
        for(;x;x >>= 1, a = mul(a, a))
            if(x & 1) res = mul(res, a);
        return res;
    }
    inline ll rho(ll n) {
        if(!(n & 1)) return 2;
        static std::mt19937_64 gen((size_t)"hehezhou");
        ll x = 0, y = 0, prod = 1;
        auto f = [&](ll o) { return mul(o, o) + 1; };
        setmod(n);
        for(int t = 30, z = 0;t % 64 || std::gcd(prod, n) == 1;++t) {
            if (x == y) x = ++ z, y = f(x);
            if(ll q = mul(prod, x + n - y)) prod = q;
            x = f(x), y = f(f(y));
        }
        return std::gcd(prod, n);
    }
    bool checkprime(ll p) {
        if(p == 1) return 0;
        setmod(p);
        ll d = __builtin_ctzll(p - 1), s = (p - 1) >> d;
        for(ll a : {2, 3, 5, 7, 11, 13, 82, 373}) {
            if(a % p == 0)
                continue;
            ll x = power(a, s), y;
            for(int i = 0;i < d;++i, x = y) {
                y = mul(x, x);
                if(y == 1 && x != 1 && x != p - 1)
                    return 0;
            }
            if(x != 1) return 0;
        }
        return 1;
    }
    std::vector<ll> get_factor(ll x) {
        std::queue<ll> q; q.push(x);
        std::vector<ll> res;
        for(;q.size();) {
            ll x = q.front(); q.pop();
            if(x == 1) continue;
            if(checkprime(x)) {
                res.push_back(x);
                continue;
            }
            ll y = rho(x);
            q.push(y), q.push(x / y);
        }
        sort(res.begin(), res.end());
        return res;
    }
}\end{lstlisting}

\subsection{分数类}
\thispagestyle{fancy}

\noindent\begin{lstlisting}
//
// Created by 墨华 on 2024/6/2.
//
template<class T>
struct Frac {
    T num;
    T den;
    Frac(T num_, T den_) : num(num_), den(den_) {
        if (den < 0) {
            den = -den;
            num = -num;
        }
    }
    Frac() : Frac(0, 1) {}
    Frac(T num_) : Frac(num_, 1) {}
    explicit operator double() const {
        return 1. * num / den;
    }
    Frac &operator+=(const Frac &rhs) {
        num = num * rhs.den + rhs.num * den;
        den *= rhs.den;
        T g = std::gcd(num, den);
        num/=g;
        den/=g;
        return *this;
    }
    Frac &operator-=(const Frac &rhs) {
        num = num * rhs.den - rhs.num * den;
        den *= rhs.den;
        T g = std::gcd(num, den);
        num/=g;
        den/=g;
        return *this;
    }
    Frac &operator*=(const Frac &rhs) {
        num *= rhs.num;
        den *= rhs.den;
        T g = std::gcd(num, den);
        num/=g;
        den/=g;
        return *this;
    }
    Frac &operator/=(const Frac &rhs) {
        num *= rhs.den;
        den *= rhs.num;
        if (den < 0) {
            num = -num;
            den = -den;
        }
        T g = std::gcd(num, den);
        num/=g;
        den/=g;
        return *this;
    }
    friend Frac operator+(Frac lhs, const Frac &rhs) {
        return lhs += rhs;
    }
    friend Frac operator-(Frac lhs, const Frac &rhs) {
        return lhs -= rhs;
    }
    friend Frac operator*(Frac lhs, const Frac &rhs) {
        return lhs *= rhs;
    }
    friend Frac operator/(Frac lhs, const Frac &rhs) {
        return lhs /= rhs;
    }
    friend Frac operator-(const Frac &a) {
        return Frac(-a.num, a.den);
    }
    friend bool operator==(const Frac &lhs, const Frac &rhs) {
        return lhs.num * rhs.den == rhs.num * lhs.den;
    }
    friend bool operator!=(const Frac &lhs, const Frac &rhs) {
        return lhs.num * rhs.den != rhs.num * lhs.den;
    }
    friend bool operator<(const Frac &lhs, const Frac &rhs) {
        return lhs.num * rhs.den < rhs.num * lhs.den;
    }
    friend bool operator>(const Frac &lhs, const Frac &rhs) {
        return lhs.num * rhs.den > rhs.num * lhs.den;
    }
    friend bool operator<=(const Frac &lhs, const Frac &rhs) {
        return lhs.num * rhs.den <= rhs.num * lhs.den;
    }
    friend bool operator>=(const Frac &lhs, const Frac &rhs) {
        return lhs.num * rhs.den >= rhs.num * lhs.den;
    }
    friend std::ostream &operator<<(std::ostream &os, Frac x) {
        T g = std::gcd(x.num, x.den);
        if (x.den == g) {
            return os << x.num / g;
        } else {
            return os << x.num / g << "/" << x.den / g;
        }
    }
};

using F = Frac<ll>;\end{lstlisting}

\subsection{大数}
\thispagestyle{fancy}

\noindent\begin{lstlisting}
const int base = 1000000000;
const int base_digits = 9; // 分解为九个数位一个数字
struct bigint {
    vector<int> a;
    int sign;

    bigint() : sign(1) {}
    bigint operator-() const {
        bigint res = *this;
        res.sign = -sign;
        return res;
    }
    bigint(long long v) {
        *this = v;
    }
    bigint(const string &s) {
        read(s);
    }
    void operator=(const bigint &v) {
        sign = v.sign;
        a = v.a;
    }
    void operator=(long long v) {
        a.clear();
        sign = 1;
        if (v < 0) sign = -1, v = -v;
        for (; v > 0; v = v / base) {
            a.push_back(v % base);
        }
    }

    // 基础加减乘除
    bigint operator+(const bigint &v) const {
        if (sign == v.sign) {
            bigint res = v;
            for (int i = 0, carry = 0; i < (int)max(a.size(), v.a.size()) || carry; ++i) {
                if (i == (int)res.a.size()) {
                    res.a.push_back(0);
                }
                res.a[i] += carry + (i < (int)a.size() ? a[i] : 0);
                carry = res.a[i] >= base;
                if (carry) {
                    res.a[i] -= base;
                }
            }
            return res;
        }
        return *this - (-v);
    }
    bigint operator-(const bigint &v) const {
        if (sign == v.sign) {
            if (abs() >= v.abs()) {
                bigint res = *this;
                for (int i = 0, carry = 0; i < (int)v.a.size() || carry; ++i) {
                    res.a[i] -= carry + (i < (int)v.a.size() ? v.a[i] : 0);
                    carry = res.a[i] < 0;
                    if (carry) {
                        res.a[i] += base;
                    }
                }
                res.trim();
                return res;
            }
            return -(v - *this);
        }
        return *this + (-v);
    }
    void operator*=(int v) {
        check(v);
        for (int i = 0, carry = 0; i < (int)a.size() || carry; ++i) {
            if (i == (int)a.size()) {
                a.push_back(0);
            }
            long long cur = a[i] * (long long)v + carry;
            carry = (int)(cur / base);
            a[i] = (int)(cur % base);
        }
        trim();
    }
    void operator/=(int v) {
        check(v);
        for (int i = (int)a.size() - 1, rem = 0; i >= 0; --i) {
            long long cur = a[i] + rem * (long long)base;
            a[i] = (int)(cur / v);
            rem = (int)(cur % v);
        }
        trim();
    }
    int operator%(int v) const {
        if (v < 0) {
            v = -v;
        }
        int m = 0;
        for (int i = a.size() - 1; i >= 0; --i) {
            m = (a[i] + m * (long long)base) % v;
        }
        return m * sign;
    }

    void operator+=(const bigint &v) {
        *this = *this + v;
    }
    void operator-=(const bigint &v) {
        *this = *this - v;
    }
    bigint operator*(int v) const {
        bigint res = *this;
        res *= v;
        return res;
    }
    bigint operator/(int v) const {
        bigint res = *this;
        res /= v;
        return res;
    }
    void operator%=(const int &v) {
        *this = *this % v;
    }

    bool operator<(const bigint &v) const {
        if (sign != v.sign) return sign < v.sign;
        if (a.size() != v.a.size()) return a.size() * sign < v.a.size() * v.sign;
        for (int i = a.size() - 1; i >= 0; i--)
            if (a[i] != v.a[i]) return a[i] * sign < v.a[i] * sign;
        return false;
    }
    bool operator>(const bigint &v) const {
        return v < *this;
    }
    bool operator<=(const bigint &v) const {
        return !(v < *this);
    }
    bool operator>=(const bigint &v) const {
        return !(*this < v);
    }
    bool operator==(const bigint &v) const {
        return !(*this < v) && !(v < *this);
    }
    bool operator!=(const bigint &v) const {
        return *this < v || v < *this;
    }

    bigint abs() const {
        bigint res = *this;
        res.sign *= res.sign;
        return res;
    }
    void check(int v) { // 检查输入的是否为负数
        if (v < 0) {
            sign = -sign;
            v = -v;
        }
    }
    void trim() { // 去除前导零
        while (!a.empty() && !a.back()) a.pop_back();
        if (a.empty()) sign = 1;
    }
    bool isZero() const { // 判断是否等于零
        return a.empty() || (a.size() == 1 && !a[0]);
    }
    friend bigint gcd(const bigint &a, const bigint &b) {
        return b.isZero() ? a : gcd(b, a % b);
    }
    friend bigint lcm(const bigint &a, const bigint &b) {
        return a / gcd(a, b) * b;
    }
    void read(const string &s) {
        sign = 1;
        a.clear();
        int pos = 0;
        while (pos < (int)s.size() && (s[pos] == '-' || s[pos] == '+')) {
            if (s[pos] == '-') sign = -sign;
            ++pos;
        }
        for (int i = s.size() - 1; i >= pos; i -= base_digits) {
            int x = 0;
            for (int j = max(pos, i - base_digits + 1); j <= i; j++) x = x * 10 + s[j] - '0';
            a.push_back(x);
        }
        trim();
    }
    friend istream &operator>>(istream &stream, bigint &v) {
        string s;
        stream >> s;
        v.read(s);
        return stream;
    }
    friend ostream &operator<<(ostream &stream, const bigint &v) {
        if (v.sign == -1) stream << '-';
        stream << (v.a.empty() ? 0 : v.a.back());
        for (int i = (int)v.a.size() - 2; i >= 0; --i)
            stream << setw(base_digits) << setfill('0') << v.a[i];
        return stream;
    }

    /* 大整数乘除大整数部分 */
    typedef vector<long long> vll;
    bigint operator*(const bigint &v) const { // 大整数乘大整数
        vector<int> a6 = convert_base(this->a, base_digits, 6);
        vector<int> b6 = convert_base(v.a, base_digits, 6);
        vll a(a6.begin(), a6.end());
        vll b(b6.begin(), b6.end());
        while (a.size() < b.size()) a.push_back(0);
        while (b.size() < a.size()) b.push_back(0);
        while (a.size() & (a.size() - 1)) a.push_back(0), b.push_back(0);
        vll c = karatsubaMultiply(a, b);
        bigint res;
        res.sign = sign * v.sign;
        for (int i = 0, carry = 0; i < (int)c.size(); i++) {
            long long cur = c[i] + carry;
            res.a.push_back((int)(cur % 1000000));
            carry = (int)(cur / 1000000);
        }
        res.a = convert_base(res.a, 6, base_digits);
        res.trim();
        return res;
    }
    friend pair<bigint, bigint> divmod(const bigint &a1,
                                       const bigint &b1) { // 大整数除大整数,同时返回答案与余数
        int norm = base / (b1.a.back() + 1);
        bigint a = a1.abs() * norm;
        bigint b = b1.abs() * norm;
        bigint q, r;
        q.a.resize(a.a.size());
        for (int i = a.a.size() - 1; i >= 0; i--) {
            r *= base;
            r += a.a[i];
            int s1 = r.a.size() <= b.a.size() ? 0 : r.a[b.a.size()];
            int s2 = r.a.size() <= b.a.size() - 1 ? 0 : r.a[b.a.size() - 1];
            int d = ((long long)base * s1 + s2) / b.a.back();
            r -= b * d;
            while (r < 0) r += b, --d;
            q.a[i] = d;
        }
        q.sign = a1.sign * b1.sign;
        r.sign = a1.sign;
        q.trim();
        r.trim();
        return make_pair(q, r / norm);
    }
    static vector<int> convert_base(const vector<int> &a, int old_digits, int new_digits) {
        vector<long long> p(max(old_digits, new_digits) + 1);
        p[0] = 1;
        for (int i = 1; i < (int)p.size(); i++) p[i] = p[i - 1] * 10;
        vector<int> res;
        long long cur = 0;
        int cur_digits = 0;
        for (int i = 0; i < (int)a.size(); i++) {
            cur += a[i] * p[cur_digits];
            cur_digits += old_digits;
            while (cur_digits >= new_digits) {
                res.push_back((int)(cur % p[new_digits]));
                cur /= p[new_digits];
                cur_digits -= new_digits;
            }
        }
        res.push_back((int)cur);
        while (!res.empty() && !res.back()) res.pop_back();
        return res;
    }
    static vll karatsubaMultiply(const vll &a, const vll &b) {
        int n = a.size();
        vll res(n + n);
        if (n <= 32) {
            for (int i = 0; i < n; i++) {
                for (int j = 0; j < n; j++) {
                    res[i + j] += a[i] * b[j];
                }
            }
            return res;
        }

        int k = n >> 1;
        vll a1(a.begin(), a.begin() + k);
        vll a2(a.begin() + k, a.end());
        vll b1(b.begin(), b.begin() + k);
        vll b2(b.begin() + k, b.end());

        vll a1b1 = karatsubaMultiply(a1, b1);
        vll a2b2 = karatsubaMultiply(a2, b2);

        for (int i = 0; i < k; i++) a2[i] += a1[i];
        for (int i = 0; i < k; i++) b2[i] += b1[i];

        vll r = karatsubaMultiply(a2, b2);
        for (int i = 0; i < (int)a1b1.size(); i++) r[i] -= a1b1[i];
        for (int i = 0; i < (int)a2b2.size(); i++) r[i] -= a2b2[i];

        for (int i = 0; i < (int)r.size(); i++) res[i + k] += r[i];
        for (int i = 0; i < (int)a1b1.size(); i++) res[i] += a1b1[i];
        for (int i = 0; i < (int)a2b2.size(); i++) res[i + n] += a2b2[i];
        return res;
    }

    void operator*=(const bigint &v) {
        *this = *this * v;
    }
    bigint operator/(const bigint &v) const {
        return divmod(*this, v).first;
    }
    void operator/=(const bigint &v) {
        *this = *this / v;
    }
    bigint operator%(const bigint &v) const {
        return divmod(*this, v).second;
    }
    void operator%=(const bigint &v) {
        *this = *this % v;
    }
};\end{lstlisting}

\subsection{拓展exgcd求逆元}
\thispagestyle{fancy}

\noindent\begin{lstlisting}
ll exgcd(ll a, ll b, ll &x, ll &y) {
    if (!b) {
        x = 1, y = 0;
        return a;
    }
    ll d = exgcd(b, a % b, y, x);
    y -= a / b * x;
    return d;
}
ll inv(ll n, ll M) {
    ll x, y;
    exgcd(n, M, x, y);
    x = (x % M + M) % M;
    return x;
}\end{lstlisting}

\subsection{线性基}
\thispagestyle{fancy}

\noindent\begin{lstlisting}
const int N=210;
const int B=60;
struct linear_basis{
    array<ll,B+1>num{};
    bool insert(ll x){
        for(int i=B-1;i>=0;i--){
            if(x&(1ll<<i)){
                if(num[i]==0){num[i]=x;
                    return true;
                };
                x^=num[i];
            }
        }
        return false;
    }
    ll querymin(ll x){
        for(int i=B-1;i>=0;i--){
            x=min(x,x^num[i]);
        }
        return x;
    }
    ll querymax(ll x){
        for(int i=B-1;i>=0;i--){
            x=max(x,x^num[i]);
        }
        return x;
    }
};\end{lstlisting}

\subsection{莫比乌斯反演}
\thispagestyle{fancy}

\noindent\begin{lstlisting}
constexpr int maxn = 1e5 + 7;
int p[maxn];
int pr[maxn];
int tot = 0;
int phi[maxn];
int mu[maxn];
void sieve(ll n) {
    p[1] = phi[1] = mu[1] = 1;
    for (ll i = 2; i < n; ++i) {
        if (!p[i]) mu[i] = -1, pr[++tot] = i, phi[i] = i - 1;
        for (int j = 1; j <= tot && i * pr[j] < n; ++j) {
            p[i * pr[j]] = pr[j];
            if (i % pr[j]) {
                mu[i * pr[j]] = -mu[i];
                phi[i * pr[j]] = phi[i] * phi[pr[j]];
            } else {
                phi[pr[j]*i] = phi[i] * pr[j];
                break;
            }
        }
    }
}
\end{lstlisting}

\section{数据结构}
\thispagestyle{fancy}

\subsection{RMQ}
\thispagestyle{fancy}

\noindent\begin{lstlisting}
template<class T,
        class Cmp = std::less<T>>
struct RMQ {
    const Cmp cmp = Cmp();
    static constexpr unsigned B = 64;
    using u64 = unsigned long long;
    int n;
    std::vector<std::vector<T>> a;
    std::vector<T> pre, suf, ini;
    std::vector<u64> stk;
    RMQ() {}
    RMQ(std::vector<T> v) {
        v.erase(v.begin(), v.begin() + 1);
        init(v);
    }
    void init(const std::vector<T>& v) {
        n = v.size();
        pre = suf = ini = v;
        stk.resize(n);
        if (!n) {
            return;
        }
        const int M = (n - 1) / B + 1;
        const int lg = std::__lg(M);
        a.assign(lg + 1, std::vector<T>(M));
        for (int i = 0; i < M; i++) {
            a[0][i] = v[i * B];
            for (int j = 1; j < B && i * B + j < n; j++) {
                a[0][i] = std::min(a[0][i], v[i * B + j], cmp);
            }
        }
        for (int i = 1; i < n; i++) {
            if (i % B) {
                pre[i] = std::min(pre[i], pre[i - 1], cmp);
            }
        }
        for (int i = n - 2; i >= 0; i--) {
            if (i % B != B - 1) {
                suf[i] = std::min(suf[i], suf[i + 1], cmp);
            }
        }
        for (int j = 0; j < lg; j++) {
            for (int i = 0; i + (2 << j) <= M; i++) {
                a[j + 1][i] = std::min(a[j][i], a[j][i + (1 << j)], cmp);
            }
        }
        for (int i = 0; i < M; i++) {
            const int l = i * B;
            const int r = std::min(1U * n, l + B);
            u64 s = 0;
            for (int j = l; j < r; j++) {
                while (s && cmp(v[j], v[std::__lg(s) + l])) {
                    s ^= 1ULL << std::__lg(s);
                }
                s |= 1ULL << (j - l);
                stk[j] = s;
            }
        }
    }
    T operator()(int l, int r) {
        --l;
        if (l / B != (r - 1) / B) {
            T ans = std::min(suf[l], pre[r - 1], cmp);
            l = l / B + 1;
            r = r / B;
            if (l < r) {
                int k = std::__lg(r - l);
                ans = std::min({ ans, a[k][l], a[k][r - (1 << k)] }, cmp);
            }
            return ans;
        }
        else {
            int x = B * (l / B);
            return ini[__builtin_ctzll(stk[r - 1] >> (l - x)) + l];
        }
    }
};\end{lstlisting}

\subsection{jls分块}
\thispagestyle{fancy}

\noindent\begin{lstlisting}
template<class T>
struct Block {
    int n;
    int B;
    std::vector<T> a;
    std::vector<T> add;
    Block(int n_) : n{n_} {
        B = std::sqrt(n);
        a.resize(n);
        add.resize((n + B - 1) / B);
    }
    void rangeAdd(int l, int r, int v) {
        r++;
        if (l >= r) {
            return;
        }
        int lb = l / B;
        int rb = (r - 1) / B;
        if (lb == rb) {
            for (int i = l; i < r; i++) {
                a[i] += v;
            }
        } else {
            for (int i = l; i < (lb + 1) * B; i++) {
                a[i] += v;
            }
            for (int i = lb + 1; i < rb; i++) {
                add[i] += v;
            }
            for (int i = rb * B; i < r; i++) {
                a[i] += v;
            }
        }
    }
    T query(int x) {
        return a[x] + add[x / B];
    }
};\end{lstlisting}

\subsection{st表}
\thispagestyle{fancy}

\noindent\begin{lstlisting}
template <class Info>
struct ST {
    int n, m;
    std::vector<std::vector<Info>> f;
    ST(std::vector<ll> &a) : n(a.size() - 1) {
        m = __lg(n) + 1;
        f.resize(m + 1, vector<Info>(n + 1));
        for (int i = 1; i <= n; i++) {
            f[0][i] = {a[i]};
        }
        for (int j = 1; j <= m; j++) {
            for (int i = 1; i + (1 << j) - 1 <= n; i++) {
                f[j][i] = f[j - 1][i] + f[j - 1][i + (1 << (j - 1))];
            }
        }
    }
    Info operator()(int l, int r) {
        int len = (r - l + 1);
        ll k = __lg(len);
        return f[k][l] + f[k][r - (1 << k) + 1];
    }
};
struct Info {
    ll x;
    friend Info operator+(const Info &a, const Info &b) {
        Info res;
        res.x = max(a.x, b.x);
        return res;
    }
};\end{lstlisting}

\subsection{zkw线段树}
\thispagestyle{fancy}

\noindent\begin{lstlisting}
template<class T>
struct Segment {
    T o[1 << 20]; int L;
    void upt(int x) {
        o[x] = o[x << 1] + o[x << 1 | 1];
    }
    void init(int n, const vector<T>&w) {
        L = 2 << std::__lg(n + 1);
        for (int i = 1; i <= n; ++i) o[i + L] = w[i];
        for (int i = L; i >= 1; --i) upt(i);
    }
    void change(int p, T v) {
        for (o[p += L] = v; p >>= 1; upt(p));
    }
    T query(int l, int r) {
        l += L - 1, r += L + 1;
        T ans{};
        for (; l ^ r ^ 1; l >>= 1, r >>= 1) {
            if ((l & 1) == 0) ans =ans+ o[l ^ 1];
            if ((r & 1) == 1) ans =ans+ o[r ^ 1];
        }
        return ans;
    }
};
struct node {
    ll val = 0;
    node () {
    }
    friend  node operator+(node lhs, node rhs) {
        node now;
        if (lhs.val > rhs.val)now = lhs;
        else now = rhs;
        return now;
    }
};
Segment<node>seg;\end{lstlisting}

\subsection{主席树}
\thispagestyle{fancy}

\noindent\begin{lstlisting}
struct Chairman_Tree {
    struct Node {int L, R, val;} tree[maxn * 500];
    void init() {
        memset(root, 0, sizeof root);
        cnt = 0;
    }
    /* 建T0空树 */
    int build(int l, int r) {
        int k = cnt++;
        tree[k].val = 0;
        if (l == r) return k;
        int mid = l + r >> 1;
        tree[k].L = build(l, mid); tree[k].R = build(mid + 1, r);
        return k;
    }
    /* 上一个版本节点P,【ppos】+=del 返回新版本节点*/
    int update (int P, int l, int r, int ppos, int del) {
        int k = cnt++;
        tree[k].val = tree[P].val + del;
        if (l == r) return k;
        int mid = l + r >> 1;
        if (ppos <= mid) {
            tree[k].L = update(tree[P].L, l, mid, ppos, del);
            tree[k].R = tree[P].R;
        } else {
            tree[k].L = tree[P].L;
            tree[k].R = update(tree[P].R, mid + 1, r, ppos, del);
        }
        return k;
    }
    int query_kth(int lt, int rt, int l, int r, int k) {
        if (l == r) return a[rk[l]];
        int mid = l + r >> 1;
        if (tree[tree[rt].L].val - tree[tree[lt].L].val >= k) return query_kth(tree[lt].L, tree[rt].L, l, mid, k);
        else return query_kth(tree[lt].R, tree[rt].R, mid + 1, r, k + tree[tree[lt].L].val - tree[tree[rt].L].val);
    }
} tree;\end{lstlisting}

\subsection{动态中位数}
\thispagestyle{fancy}

\noindent\begin{lstlisting}
struct DynamicMedian {
    multiset<ll>down,up;
    DynamicMedian() {}
    void change() {
        while (down.size() > 1 + up.size()) {
            up.insert(*down.rbegin());
            down.extract(*down.rbegin());
        }
        while (up.size() > down.size()) {
            down.insert(*up.begin());
            up.extract(*up.begin());
        }
    }
    void insert(ll x) {
        if (down.empty() || x <= *down.rbegin()) {
            down.insert(x);
        } else {
            up.insert(x);
        }
        change();
    };
    void erase(ll x){
        if(down.find(x)!=down.end()){
            down.extract(x);
        }else up.extract(x);
        change();
    }
    ll Ans() {
        return *down.rbegin();
    };
};\end{lstlisting}

\subsection{区间修改+lazy线段树}
\thispagestyle{fancy}

\noindent\begin{lstlisting}
template<class Info, class Tag>
struct LazySegmentTree {
    int n;
    std::vector<Info> info;
    std::vector<Tag> tag;
    LazySegmentTree() : n(0) {}
    LazySegmentTree(int n_) : n(n_), info(4 * n_ + 1, Info()), tag(4 * n_ + 1, Tag()) {}
    template<class T>
    LazySegmentTree(std::vector<T> init_) {
        init(init_);
    }
    template<class T>
    void init(std::vector<T> init_) {
        n = init_.size() - 1;
        info.assign(4 * n + 1, Info());
        tag.assign(4 * n + 1, Tag());
        auto build = [&](auto && build, int p, int l, int r)->void{
            if (l == r) {
                info[p] = init_[l];
                return;
            }
            int mid = (l + r) >> 1;
            build(build, p << 1, l, mid);
            build(build, p << 1 | 1, mid + 1, r);
            push_up(p);
        };
        build(build, 1, 1, n);
    }
    void push_up(int p) {
        info[p] = (info[p << 1] + info[p << 1 | 1]);
    }
    void push_down(int p) {
        info[p << 1] += tag[p];
        info[p << 1 | 1] += tag[p];
        tag[p << 1] += tag[p];
        tag[p << 1 | 1] += tag[p];
        tag[p].init();
    }
    void change(int l, int r, Tag num) {
        change(1, 1, n, l, r, num);
    }
    void change(int p, int l, int r, int nl, int nr, Tag num) {
        if (nl <= l and r <= nr) {
            info[p] += num;
            tag[p] += num;
            return;
        }
        push_down(p);
        int mid = (l + r) >> 1;
        if (nl <= mid)change(p << 1, l, mid, nl, nr, num);
        if (nr > mid)change(p << 1 | 1, mid + 1, r, nl, nr, num);
        push_up(p);
    }
    Info query(int l, int r) {
        return query(1, 1, n, l, r);
    }
    Info query(int p, int l, int r, int nl, int nr) {
        if (nl <= l and r <= nr) {
            return info[p];
        }
        push_down(p);
        push_up(p);
        int mid = (l + r) >> 1;
        if (nr <= mid)return query(p << 1, l, mid, nl, nr);
        if (nl > mid)return query(p << 1 | 1, mid + 1, r, nl, nr);
        return query(p << 1, l, mid, nl, nr) + query(p << 1 | 1, mid + 1, r, nl, nr);
    }
    int find_first(int p, int l, int r, const std::function<bool(const Info &)> &f)
    {
        if (l == r) {
            return l;
        }
        push_down(p);
        push_up(p);
        int mid = (l + r) >> 1;
        if (f(info[p << 1]))return find_first(p << 1, l, mid, f);
        else return find_first(p << 1 | 1, mid + 1, r, f);
    }
    int find_first(const std::function<bool(const Info &)> &f) {
        return find_first(1, 1, n, f);
    }
};
struct tag {
    void init() {
    
    }
    tag&operator+=(const tag &t) & {
    
        return *this;
    }
};
struct node {
    friend node operator+(node lhs, node rhs) {
        node res;
        
        return res;
    }
    node&operator+=(tag t) {
        
        return *this;
    }
};\end{lstlisting}

\subsection{单点修改线段树}
\thispagestyle{fancy}

\noindent\begin{lstlisting}
template<class Info>
struct Segment {
    std::vector<Info> info;
    int n;
    Segment() : n(0) {}
    Segment(int n) : n(n), info(4*n+1) {}
    Segment(int n, vector<Info>&v) : n(n), info(4*n+1) {
        init(v);
    }
    void pushup(int p) {
        info[p] = info[p << 1] + info[p << 1 | 1];
    }
    void init(vector<Info>&v) {
        auto build = [&](auto && build, int p, int l, int r) {
            if (l == r) {
                info[p] = v[l];
                return;
            }
            int mid = (l + r) >> 1;
            build(build, p << 1, l, mid);
            build(build, p << 1 | 1, mid + 1, r);
            pushup(p);
        };
        build(build, 1, 1, n);
    }
    void change(int pos, Info x) {
        change(1, 1, n, pos, x);
    }
    void change(int p, int l, int r, int pos, Info x) {
        if (l == r) {
            info[p] = x;
            return;
        }
        int mid = (l + r) >> 1;
        if (pos <= mid)change(p << 1, l, mid, pos, x);
        else change(p << 1 | 1, mid + 1, r, pos, x);
        pushup(p);
    }
    Info query(int l, int r) {
        return query(1, 1, n, l, r);
    }
    Info query(int p, int l, int r, int nl, int nr) {
        if (nl <= l and r <= nr)return info[p];
        int mid = (l + r) >> 1;
        if (nr <= mid)return query(p << 1, l, mid, nl, nr);
        if (nl > mid)return query(p << 1 | 1, mid + 1, r, nl, nr);
        return query(p << 1, l, mid, nl, nr) + query(p << 1 | 1, mid + 1, r, nl, nr);
    }
    int find_first(int p, int l, int r, const std::function<bool(const Info &)> &f)
    {
        if (l == r) {
            return l;
        }
        int mid = (l + r) >> 1;
        if (f(info[p << 1]))return find_first(p << 1, l, mid, f);
        else return find_first(p << 1 | 1, mid + 1, r, f);
    }
};
struct node {
    friend  node operator+(node lhs, node rhs) {
        node now;
        if (lhs.val < rhs.val)now = lhs;
        else now = rhs;
        return now;
    }
};\end{lstlisting}

\subsection{扫描线node版}
\thispagestyle{fancy}

\noindent\begin{lstlisting}
struct tag {
    ll val = 0;
    void init() {
        val = 0;
    }
    tag&operator+=(const tag &t) & {
        val += t.val;
        return *this;
    }
};
struct node {
    int minv, mincnt;
    friend node operator+(const node &l, const node &r) {
        node a;
        a.minv = min(l.minv, r.minv);
        if (l.minv == r.minv)a.mincnt = l.mincnt + r.mincnt;
        else if (l.minv < r.minv)a.mincnt = l.mincnt;
        else a.mincnt = r.mincnt;
        return a;
    }
    node&operator+=(tag&t) {
        minv += t.val;
        return *this;
    }
};\end{lstlisting}

\subsection{树状数组}
\thispagestyle{fancy}

\noindent\begin{lstlisting}
template <typename T>
struct BIT {
    int n;
    std::vector<T> a;

    BIT(int n_ = 0) {
        init(n_);
    }

    void init(int n_) {
        n = n_;
        a.assign(n + 2, T{});
    }

    void add(int x, const T &v) {
        for (int i = x + 1; i <= n; i += i & -i) {
            a[i - 1] = a[i - 1] + v;
        }
    }

    T sum(int x) {
        T ans{};
        x++;
        for (int i = x; i > 0; i -= i & -i) {
            ans = ans + a[i - 1];
        }
        return ans;
    }

    T getSum(int l, int r) {
        if (l > r)swap(l, r);
        return sum(r) - sum(l - 1);
    }
    void modify(int l, int r, T x) {
        if (l > r)swap(l, r);
        add(l, x);
        add(r + 1, -x);
    }
    int select(const T &k) {
        int x = 0;
        T cur{};
        for (int i = 1 << std::__lg(n); i; i /= 2) {
            if (x + i <= n && cur + a[x + i - 1] <= k) {
                x += i;
                cur = cur + a[x - 1];
            }
        }
        return x;
    }
};\end{lstlisting}

\subsection{树链剖分}
\thispagestyle{fancy}

\noindent\begin{lstlisting}
template<class SegmentTree, class Info>
struct Trh {
    std::vector<int> sz, top, dep, parent, in, out;
    int cur, n;
    SegmentTree seg;
    std::vector<std::vector<int>> e;
    Trh(int _n) : n(_n), sz(_n + 1), top(_n + 1), dep(_n + 1), parent(_n + 1, -1), e(_n + 1), in(_n + 1), cur(0), out(_n + 1) {
        seg.init(_n);
    }
    void addEdge(int u, int v) {
        e[u].push_back(v);
        e[v].push_back(u);
    }
    void init(int s) {
        dfsSz(s);
        dfsHLD(s);
    }
    void dfsSz(int u) {
        if (parent[u] != -1)
            e[u].erase(std::find(e[u].begin(), e[u].end(), parent[u]));
        sz[u] = 1;
        for (int &v : e[u]) {
            parent[v] = u;
            dep[v] = dep[u] + 1;
            dfsSz(v);
            sz[u] += sz[v];
            if (sz[v] > sz[e[u][0]])
                std::swap(v, e[u][0]);
        }
    }
    void dfsHLD(int u) {
        in[u] = ++cur;
        for (int v : e[u]) {
            if (v == e[u][0]) {
                top[v] = top[u];
            } else {
                top[v] = v;
            }
            dfsHLD(v);
        }
        out[u] = cur;
    }
    int lca(int u, int v) {
        while (top[u] != top[v]) {
            if (dep[top[u]] > dep[top[v]]) {
                u = parent[top[u]];
            } else {
                v = parent[top[v]];
            }
        }
        if (dep[u] < dep[v]) {
            return u;
        } else {
            return v;
        }
    }
    void change(int u, int v, ll add) {
        tag p{add};
        while (top[u] != top[v]) {
            if (dep[top[u]] > dep[top[v]]) {
                seg.change(1, 1, n, in[top[u]], in[u], p);
                u = parent[top[u]];
            } else {
                seg.change(1, 1, n, in[top[v]], in[v], p);
                v = parent[top[v]];
            }
        }
        if (dep[u] < dep[v]) {
            seg.change(1, 1, n, in[u], in[v], p);
        } else {
            seg.change(1, 1, n, in[v], in[u], p);
        }
    }
    node query(int u, int v) {
        node ans;
        while (top[u] != top[v]) {
            if (dep[top[u]] > dep[top[v]]) {
                ans = ans + seg.query(1, 1, n, in[top[u]], in[u]);
                u = parent[top[u]];
            } else {
                ans = ans + seg.query(1, 1, n, in[top[v]], in[v]);
                v = parent[top[v]];
            }
        }
        if (dep[u] < dep[v]) {
            ans = ans + seg.query(1, 1, n, in[u], in[v]);
        } else {
            ans = ans + seg.query(1, 1, n, in[v], in[u]);
        }
        return ans;
    }
};\end{lstlisting}

\subsection{珂朵莉树}
\thispagestyle{fancy}

\noindent\begin{lstlisting}
struct ODT {
    struct odt {
        int l, r;
        mutable int x;
        bool operator < (const odt &a) const {
            return l < a.l;
        }
    };
    set<odt> tr;
    typedef set <odt> :: iterator IT;
    ODT(int l, int r, int x) {
        tr.insert({l, r, x});
    }
    IT split(int pos) { //将pos-1和pos之间切开,返回pos所在区间指针
        auto it = tr.lower_bound({pos, 0, 0});
        if (it != tr.end() && it->l == pos) return it;
        it--;
        int l = it->l, r = it->r, x = it->x;
        tr.erase(it);
        tr.insert({l, pos - 1, x});
        return tr.insert({pos, r, x}).first;
    }
    void assign(int l, int r, int x) {
        auto R = split(r + 1);
        auto L = split(l);
        tr.erase(L, R);
        tr.insert({l, r, x});
    }
    void modify(int l, int r) {
        auto R = split(r + 1);
        auto L = split(l);
        for (auto it = L; it != R; it++) {
            // 对it->x暴力修改
        }
    }
    int query() {
        int ans = 0;
        for (auto it = tr.begin(); it != tr.end(); it++) {

        }
        return ans;
    }
};\end{lstlisting}

\subsection{笛卡尔树}
\thispagestyle{fancy}

\noindent\begin{lstlisting}
template<class T>
struct CaT{
    vector<T>l,r;
    CaT() {}
    CaT(std::vector<T>&v,int n) {
        l.assign(n+1,-1);
        r.assign(n+1,-1);
        build(v,n);
    }
    void build(std::vector<T>&v,int n) {
        vector<int> st;
        int root = 0;
        for (int i = 1; i <= n; i++) {
            int last = -1;
            while (!st.empty() && v[st.back()] > v[i]) {
                last = st.back();
                st.pop_back();
            }
            if (!st.empty())r[st.back()] = i;
            else root = i;
            l[i] = last;
            st.push_back(i);
        }
    }
};\end{lstlisting}

\subsection{线段树代替set}
\thispagestyle{fancy}

\noindent\begin{lstlisting}
struct node {
    ll val;
    int id;
    node () {
    }
    friend  node operator+(node lhs, node rhs) {
        node now;
        if (lhs.val < rhs.val)now = lhs;
        else now = rhs;
        return now;
    }
};
struct Segment {
    std::vector<node> tre;
    int n;
    Segment() : n(0) {}
    Segment(int n) : n(n), tre(4 << std::__lg(n)+2) {}
    Segment(int n, vector<node>&v) : n(n), tre(4 << std::__lg(n)+2) {
        init(v);
    }
    void pushup(int p) {
        tre[p] = tre[p << 1] + tre[p << 1 | 1];
    }
    void init(vector<node>&v) {
        auto build = [&](auto && build, int p, int l, int r) {
            if (l == r) {
                tre[p] = v[l];
                return;
            }
            int mid = (l + r) >> 1;
            build(build, p << 1, l, mid);
            build(build, p << 1 | 1, mid + 1, r);
            pushup(p);
        };
        build(build, 1, 1, n);
    }
    void change(int p, int l, int r, int pos, ll x) {
        if (l == r) {
            tre[p].val = x;
            return;
        }
        int mid = (l + r) >> 1;
        if (pos <= mid)change(p << 1, l, mid, pos, x);
        else change(p << 1 | 1, mid + 1, r, pos, x);
        pushup(p);
    }
    node query(int p, int l, int r, int nl, int nr) {
        if (nl <= l and r <= nr)return tre[p];
        int mid = (l + r) >> 1;
        if (nr <= mid)return query(p << 1, l, mid, nl, nr);
        if (nl > mid)return query(p << 1 | 1, mid + 1, r, nl, nr);
        return query(p << 1, l, mid, nl, nr) + query(p << 1 | 1, mid + 1, r, nl, nr);
    }
    void erase(int u) {
        change(1, 1, n, u, LNF);
    }
    void insert(int u,ll val){
        change(1,1,n,u,node{val,u});
    }
};\end{lstlisting}

\subsection{莫队}
\thispagestyle{fancy}

\noindent\begin{lstlisting}
#include<bits/stdc++.h>
using namespace std;
typedef long long ll;
const int maxn=2e5+7;
ll tmp=0;
int main(){
    int n,m;
    cin>>n>>m;
    vector<int>v(n+1);
    for(int i=1;i<=n;i++)cin>>v[i];
    vector<ll>ans(m);
    std::vector<std::array<int, 3>> que(m);
    for(int i=0;i<m;i++){
        int l,r;
        cin>>l>>r;
        que[i]={l,r,i};
    }
    const int B=400;
     vector<int>cnt(maxn,0);
    std::sort(que.begin(),que.end(),[&](array<int,3>a, array<int,3>b) {
        if (a[0] / B != b[0] / B) {
            return a[0]/B < b[0]/B;
        } else {
            return a[1] < b[1];
        }
    });
    
    ll res=0;
    auto add = [&](int x) {
        x=v[x];
        res += 1LL * cnt[x] * (cnt[x] - 1) / 2;
        cnt[x] += 1;
    };
    auto del = [&](int x) {
        x=v[x];
        cnt[x] -= 1;
        res -= 1LL * cnt[x] * (cnt[x] - 1) / 2;
    };
    int l = 1, r = 0;
    for(int i=0;i<m;i++){
        while(r<que[i][1])r++,add(r);
        while(l>que[i][0])l--,add(l);
        while(r>que[i][1])del(r),r--;
        while(l<que[i][0])del(l),l++;
        ans[que[i][2]]=res;
    }
    for(int i=0;i<m;i++)cout<<ans[i]<<"\n";
}\end{lstlisting}

\section{计算几何}
\thispagestyle{fancy}

\subsection{牛客计算几何}
\thispagestyle{fancy}

\noindent\begin{lstlisting}
using point_t = long long;
// using point_t = long double; //全局数据类型

constexpr point_t eps = 1e-8;
constexpr point_t INF = numeric_limits<point_t>::max();
constexpr long double PI = 3.1415926535897932384l;

// 点与向量
template<typename T> struct point
{
    T x, y;

    bool operator==(const point &a) const {return (abs(x - a.x) <= eps && abs(y - a.y) <= eps);}
    bool operator<(const point &a) const {if (abs(x - a.x) <= eps) return y < a.y - eps; return x < a.x - eps;}
    bool operator>(const point &a) const {return !(*this < a || *this == a);}
    point operator+(const point &a) const {return {x + a.x, y + a.y};}
    point operator-(const point &a) const {return {x - a.x, y - a.y};}
    point operator-() const {return { -x, -y};}
    point operator*(const T k) const {return {k * x, k * y};}
    point operator/(const T k) const {return {x / k, y / k};}
    T operator*(const point &a) const {return x * a.x + y * a.y;} // 点积
    T operator^(const point &a) const {return x * a.y - y * a.x;} // 叉积,注意优先级
    int toleft(const point &a) const {const auto t = (*this)^a; return (t > eps) - (t < -eps);} // to-left 测试
    T len2() const {return (*this) * (*this);} // 向量长度的平方
    T dis2(const point &a) const {return (a - (*this)).len2();} // 两点距离的平方

    // 涉及浮点数
    long double len() const {return sqrtl(len2());}  // 向量长度
    long double dis(const point &a) const {return sqrtl(dis2(a));}  // 两点距离
    long double ang(const point &a) const {return acosl(max(-1.0l, min(1.0l, ((*this) * a) / (len() * a.len()))));} // 向量夹角
    point rot(const long double rad) const {return {x * cos(rad) - y * sin(rad), x * sin(rad) + y * cos(rad)};} // 逆时针旋转(给定角度)
    point rot(const long double cosr, const long double sinr) const {return {x*cosr - y * sinr, x*sinr + y * cosr};} // 逆时针旋转(给定角度的正弦与余弦)
    friend constexpr std::ostream &operator<<(std::ostream &os, const point &a) {
        return os << a.x << " " << a.y;
    }
    friend constexpr std::istream &operator>>(std::istream &is, point &a) {
        is >> a.x >> a.y;
        return is;
    }
};
using Point = point<point_t>;

// 极角排序
struct argcmp
{
    bool operator()(const Point &a, const Point &b) const
    {
        const auto quad = [](const Point & a)
        {
            if (a.y < -eps) return 1;
            if (a.y > eps) return 4;
            if (a.x < -eps) return 5;
            if (a.x > eps) return 3;
            return 2;
        };
        const int qa = quad(a), qb = quad(b);
        if (qa != qb) return qa < qb;
        const auto t = a ^ b;
        // if (abs(t)<=eps) return a*a<b*b-eps;  // 不同长度的向量需要分开
        return t > eps;
    }
};

// 直线
template<typename T> struct line
{
    point<T> p, v; // p 为直线上一点,v 为方向向量

    bool operator==(const line &a) const {return v.toleft(a.v) == 0 && v.toleft(p - a.p) == 0;}
    int toleft(const point<T> &a) const {return v.toleft(a - p);} // to-left 测试
    bool operator<(const line &a) const  // 半平面交算法定义的排序
    {
        if (abs(v ^ a.v) <= eps && v * a.v >= -eps) return toleft(a.p) == -1;
        return argcmp()(v, a.v);
    }

    // 涉及浮点数
    point<T> inter(const line &a) const {return p + v * ((a.v ^ (p - a.p)) / (v ^ a.v));} // 直线交点
    long double dis(const point<T> &a) const {return abs(v ^ (a - p)) / v.len();} // 点到直线距离
    point<T> proj(const point<T> &a) const {return p + v * ((v * (a - p)) / (v * v));} // 点在直线上的投影
};

using Line = line<point_t>;

//线段
template<typename T> struct segment
{
    point<T> a, b;

    bool operator<(const segment &s) const {return make_pair(a, b) < make_pair(s.a, s.b);}

    // 判定性函数建议在整数域使用

    // 判断点是否在线段上
    // -1 点在线段端点 | 0 点不在线段上 | 1 点严格在线段上
    int is_on(const point<T> &p) const
    {
        if (p == a || p == b) return -1;
        return (p - a).toleft(p - b) == 0 && (p - a) * (p - b) < -eps;
    }

    // 判断线段直线是否相交
    // -1 直线经过线段端点 | 0 线段和直线不相交 | 1 线段和直线严格相交
    int is_inter(const line<T> &l) const
    {
        if (l.toleft(a) == 0 || l.toleft(b) == 0) return -1;
        return l.toleft(a) != l.toleft(b);
    }

    // 判断两线段是否相交
    // -1 在某一线段端点处相交 | 0 两线段不相交 | 1 两线段严格相交
    int is_inter(const segment<T> &s) const
    {
        if (is_on(s.a) || is_on(s.b) || s.is_on(a) || s.is_on(b)) return -1;
        const line<T> l{a, b - a}, ls{s.a, s.b - s.a};
        return l.toleft(s.a) * l.toleft(s.b) == -1 && ls.toleft(a) * ls.toleft(b) == -1;
    }

    // 点到线段距离
    long double dis(const point<T> &p) const
    {
        if ((p - a) * (b - a) < -eps || (p - b) * (a - b) < -eps) return min(p.dis(a), p.dis(b));
        const line<T> l{a, b - a};
        return l.dis(p);
    }

    // 两线段间距离
    long double dis(const segment<T> &s) const
    {
        if (is_inter(s)) return 0;
        return min({dis(s.a), dis(s.b), s.dis(a), s.dis(b)});
    }
};

using Segment = segment<point_t>;

// 多边形
template<typename T> struct polygon
{
    vector<point<T>> p;  // 以逆时针顺序存储

    size_t nxt(const size_t i) const {return i == p.size() - 1 ? 0 : i + 1;}
    size_t pre(const size_t i) const {return i == 0 ? p.size() - 1 : i - 1;}

    // 回转数
    // 返回值第一项表示点是否在多边形边上
    // 对于狭义多边形,回转数为 0 表示点在多边形外,否则点在多边形内
    pair<bool, int> winding(const point<T> &a) const
    {
        int cnt = 0;
        for (size_t i = 0; i < p.size(); i++)
        {
            const point<T> u = p[i], v = p[nxt(i)];
            if (abs((a - u) ^ (a - v)) <= eps && (a - u) * (a - v) <= eps) return {true, 0};
            if (abs(u.y - v.y) <= eps) continue;
            const Line uv = {u, v - u};
            if (u.y < v.y - eps && uv.toleft(a) <= 0) continue;
            if (u.y > v.y + eps && uv.toleft(a) >= 0) continue;
            if (u.y < a.y - eps && v.y >= a.y - eps) cnt++;
            if (u.y >= a.y - eps && v.y < a.y - eps) cnt--;
        }
        return {false, cnt};
    }

    // 多边形面积的两倍
    // 可用于判断点的存储顺序是顺时针或逆时针
    T area() const
    {
        T sum = 0;
        for (size_t i = 0; i < p.size(); i++) sum += p[i] ^ p[nxt(i)];
        return sum;
    }

    // 多边形的周长
    long double circ() const
    {
        long double sum = 0;
        for (size_t i = 0; i < p.size(); i++) sum += p[i].dis(p[nxt(i)]);
        return sum;
    }
};

using Polygon = polygon<point_t>;

//凸多边形
template<typename T> struct convex: polygon<T>
{
    // 闵可夫斯基和
    convex operator+(const convex &c) const
    {
        const auto &p = this->p;
        vector<Segment> e1(p.size()), e2(c.p.size()), edge(p.size() + c.p.size());
        vector<point<T>> res; res.reserve(p.size() + c.p.size());
        const auto cmp = [](const Segment & u, const Segment & v) {return argcmp()(u.b - u.a, v.b - v.a);};
        for (size_t i = 0; i < p.size(); i++) e1[i] = {p[i], p[this->nxt(i)]};
        for (size_t i = 0; i < c.p.size(); i++) e2[i] = {c.p[i], c.p[c.nxt(i)]};
        rotate(e1.begin(), min_element(e1.begin(), e1.end(), cmp), e1.end());
        rotate(e2.begin(), min_element(e2.begin(), e2.end(), cmp), e2.end());
        merge(e1.begin(), e1.end(), e2.begin(), e2.end(), edge.begin(), cmp);
        const auto check = [](const vector<point<T>> &res, const point<T> &u)
        {
            const auto back1 = res.back(), back2 = *prev(res.end(), 2);
            return (back1 - back2).toleft(u - back1) == 0 && (back1 - back2) * (u - back1) >= -eps;
        };
        auto u = e1[0].a + e2[0].a;
        for (const auto &v : edge)
        {
            while (res.size() > 1 && check(res, u)) res.pop_back();
            res.push_back(u);
            u = u + v.b - v.a;
        }
        if (res.size() > 1 && check(res, res[0])) res.pop_back();
        return {res};
    }

    // 旋转卡壳
    // 例:凸多边形的直径的平方
    T rotcaliper() const
    {
        const auto &p = this->p;
        if (p.size() == 1) return 0;
        if (p.size() == 2) return p[0].dis2(p[1]);
        const auto area = [](const point<T> &u, const point<T> &v, const point<T> &w) {return (w - u) ^ (w - v);};
        T ans = 0;
        for (size_t i = 0, j = 1; i < p.size(); i++)
        {
            const auto nxti = this->nxt(i);
            ans = max({ans, p[j].dis2(p[i]), p[j].dis2(p[nxti])});
            while (area(p[this->nxt(j)], p[i], p[nxti]) >= area(p[j], p[i], p[nxti]))
            {
                j = this->nxt(j);
                ans = max({ans, p[j].dis2(p[i]), p[j].dis2(p[nxti])});
            }
        }
        return ans;
    }

    // 判断点是否在凸多边形内
    // 复杂度 O(logn)
    // -1 点在多边形边上 | 0 点在多边形外 | 1 点在多边形内
    int is_in(const point<T> &a) const
    {
        const auto &p = this->p;
        if (p.size() == 1) return a == p[0] ? -1 : 0;
        if (p.size() == 2) return segment<T> {p[0], p[1]} .is_on(a) ? -1 : 0;
        if (a == p[0]) return -1;
        if ((p[1] - p[0]).toleft(a - p[0]) == -1 || (p.back() - p[0]).toleft(a - p[0]) == 1) return 0;
        const auto cmp = [&](const point<T> &u, const point<T> &v) {return (u - p[0]).toleft(v - p[0]) == 1;};
        const size_t i = lower_bound(p.begin() + 1, p.end(), a, cmp) - p.begin();
        if (i == 1) return segment<T> {p[0], p[i]} .is_on(a) ? -1 : 0;
        if (i == p.size() - 1 && segment<T> {p[0], p[i]} .is_on(a)) return -1;
        if (segment<T> {p[i - 1], p[i]} .is_on(a)) return -1;
        return (p[i] - p[i - 1]).toleft(a - p[i - 1]) > 0;
    }

    // 凸多边形关于某一方向的极点
    // 复杂度 O(logn)
    // 参考资料:https://codeforces.com/blog/entry/48868
    template<typename F> size_t extreme(const F &dir) const
    {
        const auto &p = this->p;
        const auto check = [&](const size_t i) {return dir(p[i]).toleft(p[this->nxt(i)] - p[i]) >= 0;};
        const auto dir0 = dir(p[0]); const auto check0 = check(0);
        if (!check0 && check(p.size() - 1)) return 0;
        const auto cmp = [&](const point<T> &v)
        {
            const size_t vi = &v - p.data();
            if (vi == 0) return 1;
            const auto checkv = check(vi);
            const auto t = dir0.toleft(v - p[0]);
            if (vi == 1 && checkv == check0 && t == 0) return 1;
            return checkv ^ (checkv == check0 && t <= 0);
        };
        return partition_point(p.begin(), p.end(), cmp) - p.begin();
    }

    // 过凸多边形外一点求凸多边形的切线,返回切点下标
    // 复杂度 O(logn)
    // 必须保证点在多边形外
    pair<size_t, size_t> tangent(const point<T> &a) const
    {
        const size_t i = extreme([&](const point<T> &u) {return u - a;});
        const size_t j = extreme([&](const point<T> &u) {return a - u;});
        return {i, j};
    }

    // 求平行于给定直线的凸多边形的切线,返回切点下标
    // 复杂度 O(logn)
    pair<size_t, size_t> tangent(const line<T> &a) const
    {
        const size_t i = extreme([&](...) {return a.v;});
        const size_t j = extreme([&](...) {return -a.v;});
        return {i, j};
    }
};

using Convex = convex<point_t>;
// 点集的凸包
// Andrew 算法,复杂度 O(nlogn)
Convex convexhull(vector<Point> p)
{
    vector<Point> st;
    if (p.empty()) return Convex{st};
    sort(p.begin(), p.end());
    const auto check = [](const vector<Point> &st, const Point & u)
    {
        const auto back1 = st.back(), back2 = *prev(st.end(), 2);
        return (back1 - back2).toleft(u - back1) <= 0;
    };
    for (const Point &u : p)
    {
        while (st.size() > 1 && check(st, u)) st.pop_back();
        st.push_back(u);
    }
    size_t k = st.size();
    p.pop_back(); reverse(p.begin(), p.end());
    for (const Point &u : p)
    {
        while (st.size() > k && check(st, u)) st.pop_back();
        st.push_back(u);
    }
    st.pop_back();
    return Convex{st};
}
// 圆
template<class T>
struct Circle
{
    Point c;
    T r;

    bool operator==(const Circle &a) const {return c == a.c && abs(r - a.r) <= eps;}
    long double circ() const {return 2 * PI * r;} // 周长
    long double area() const {return PI * r * r;} // 面积

    // 点与圆的关系
    // -1 圆上 | 0 圆外 | 1 圆内
    int is_in(const Point &p) const {
        const __int128 d = p.dis2(c);
        if (d < r * r)return -1;
        return (d == r * r);
    }
    // 直线与圆关系
    // 0 相离 | 1 相切 | 2 相交
    int relation(const Line &l) const
    {
        __int128 d = (l.v ^ (c - l.p));
        __int128 len = l.v.len2();
        if (d * d < (len * r * r))return 2;
        return (d * d > (len * r * r));
    }

    // 圆与圆关系
    // -1 相同 | 0 相离 | 1 外切 | 2 相交 | 3 内切 | 4 内含
    int relation(const Circle &a) const
    {
        if (*this == a) return -1;
        const long double d = c.dis(a.c);
        if (d > r + a.r + eps) return 0;
        if (abs(d - r - a.r) <= eps) return 1;
        if (abs(d - abs(r - a.r)) <= eps) return 3;
        if (d < abs(r - a.r) - eps) return 4;
        return 2;
    }

    // 直线与圆的交点
    vector<Point> inter(const Line &l) const
    {
        const long double d = l.dis(c);
        const Point p = l.proj(c);
        const int t = relation(l);
        if (t == 0) return vector<Point>();
        if (t == 1) return vector<Point> {p};
        const long double k = sqrt(r * r - d * d);
        return vector<Point> {p - (l.v / l.v.len())*k, p + (l.v / l.v.len())*k};
    }

    // 圆与圆交点
    vector<Point> inter(const Circle &a) const
    {
        const long double d = c.dis(a.c);
        const int t = relation(a);
        if (t == -1 || t == 0 || t == 4) return vector<Point>();
        Point e = a.c - c; e = e / e.len() * r;
        if (t == 1 || t == 3)
        {
            if (r * r + d * d - a.r * a.r >= -eps) return vector<Point> {c + e};
            return vector<Point> {c - e};
        }
        const long double costh = (r * r + d * d - a.r * a.r) / (2 * r * d), sinth = sqrt(1 - costh * costh);
        return vector<Point> {c + e.rot(costh, -sinth), c + e.rot(costh, sinth)};
    }

    // 圆与圆交面积
    long double inter_area(const Circle &a) const
    {
        const long double d = c.dis(a.c);
        const int t = relation(a);
        if (t == -1) return area();
        if (t < 2) return 0;
        if (t > 2) return min(area(), a.area());
        const long double costh1 = (r * r + d * d - a.r * a.r) / (2 * r * d), costh2 = (a.r * a.r + d * d - r * r) / (2 * a.r * d);
        const long double sinth1 = sqrt(1 - costh1 * costh1), sinth2 = sqrt(1 - costh2 * costh2);
        const long double th1 = acos(costh1), th2 = acos(costh2);
        return r * r * (th1 - costh1 * sinth1) + a.r * a.r * (th2 - costh2 * sinth2);
    }

    // 过圆外一点圆的切线
    vector<Line> tangent(const Point &a) const
    {
        const int t = is_in(a);
        if (t == 1) return vector<Line>();
        if (t == -1)
        {
            const Point v = { -(a - c).y, (a - c).x};
            return vector<Line> {{a, v}};
        }
        Point e = a - c; e = e / e.len() * r;
        const long double costh = r / c.dis(a), sinth = sqrt(1 - costh * costh);
        const Point t1 = c + e.rot(costh, -sinth), t2 = c + e.rot(costh, sinth);
        return vector<Line> {{a, t1 - a}, {a, t2 - a}};
    }

    // 两圆的公切线
    vector<Line> tangent(const Circle &a) const
    {
        const int t = relation(a);
        vector<Line> lines;
        if (t == -1 || t == 4) return lines;
        if (t == 1 || t == 3)
        {
            const Point p = inter(a)[0], v = { -(a.c - c).y, (a.c - c).x};
            lines.push_back({p, v});
        }
        const long double d = c.dis(a.c);
        const Point e = (a.c - c) / (a.c - c).len();
        if (t <= 2)
        {
            const long double costh = (r - a.r) / d, sinth = sqrt(1 - costh * costh);
            const Point d1 = e.rot(costh, -sinth), d2 = e.rot(costh, sinth);
            const Point u1 = c + d1 * r, u2 = c + d2 * r, v1 = a.c + d1 * a.r, v2 = a.c + d2 * a.r;
            lines.push_back({u1, v1 - u1}); lines.push_back({u2, v2 - u2});
        }
        if (t == 0)
        {
            const long double costh = (r + a.r) / d, sinth = sqrt(1 - costh * costh);
            const Point d1 = e.rot(costh, -sinth), d2 = e.rot(costh, sinth);
            const Point u1 = c + d1 * r, u2 = c + d2 * r, v1 = a.c - d1 * a.r, v2 = a.c - d2 * a.r;
            lines.push_back({u1, v1 - u1}); lines.push_back({u2, v2 - u2});
        }
        return lines;
    }

    // 圆的反演
    tuple<int, Circle, Line> inverse(const Line &l) const
    {
        const Circle null_c = {{0.0, 0.0}, 0.0};
        const Line null_l = {{0.0, 0.0}, {0.0, 0.0}};
        if (l.toleft(c) == 0) return {2, null_c, l};
        const Point v = l.toleft(c) == 1 ? Point{l.v.y, -l.v.x}: Point{ -l.v.y, l.v.x};
        const long double d = r * r / l.dis(c);
        const Point p = c + v / v.len() * d;
        return {1, {(c + p) / 2, d / 2}, null_l};
    }

    tuple<int, Circle, Line> inverse(const Circle &a) const
    {
        const Circle null_c = {{0.0, 0.0}, 0.0};
        const Line null_l = {{0.0, 0.0}, {0.0, 0.0}};
        const Point v = a.c - c;
        if (a.is_in(c) == -1)
        {
            const long double d = r * r / (a.r + a.r);
            const Point p = c + v / v.len() * d;
            return {2, null_c, {p, { -v.y, v.x}}};
        }
        if (c == a.c) return {1, {c, r*r / a.r}, null_l};
        const long double d1 = r * r / (c.dis(a.c) - a.r), d2 = r * r / (c.dis(a.c) + a.r);
        const Point p = c + v / v.len() * d1, q = c + v / v.len() * d2;
        return {1, {(p + q) / 2, p.dis(q) / 2}, null_l};
    }
};

// 圆与多边形面积交
long double area_inter(const Circle &circ, const Polygon &poly)
{
    const auto cal = [](const Circle & circ, const Point & a, const Point & b)
    {
        if ((a - circ.c).toleft(b - circ.c) == 0) return 0.0l;
        const auto ina = circ.is_in(a), inb = circ.is_in(b);
        const Line ab = {a, b - a};
        if (ina && inb) return ((a - circ.c) ^ (b - circ.c)) / 2;
        if (ina && !inb)
        {
            const auto t = circ.inter(ab);
            const Point p = t.size() == 1 ? t[0] : t[1];
            const long double ans = ((a - circ.c) ^ (p - circ.c)) / 2;
            const long double th = (p - circ.c).ang(b - circ.c);
            const long double d = circ.r * circ.r * th / 2;
            if ((a - circ.c).toleft(b - circ.c) == 1) return ans + d;
            return ans - d;
        }
        if (!ina && inb)
        {
            const Point p = circ.inter(ab)[0];
            const long double ans = ((p - circ.c) ^ (b - circ.c)) / 2;
            const long double th = (a - circ.c).ang(p - circ.c);
            const long double d = circ.r * circ.r * th / 2;
            if ((a - circ.c).toleft(b - circ.c) == 1) return ans + d;
            return ans - d;
        }
        const auto p = circ.inter(ab);
        if (p.size() == 2 && Segment{a, b} .dis(circ.c) <= circ.r + eps)
        {
            const long double ans = ((p[0] - circ.c) ^ (p[1] - circ.c)) / 2;
            const long double th1 = (a - circ.c).ang(p[0] - circ.c), th2 = (b - circ.c).ang(p[1] - circ.c);
            const long double d1 = circ.r * circ.r * th1 / 2, d2 = circ.r * circ.r * th2 / 2;
            if ((a - circ.c).toleft(b - circ.c) == 1) return ans + d1 + d2;
            return ans - d1 - d2;
        }
        const long double th = (a - circ.c).ang(b - circ.c);
        if ((a - circ.c).toleft(b - circ.c) == 1) return circ.r * circ.r * th / 2;
        return -circ.r * circ.r * th / 2;
    };

    long double ans = 0;
    for (size_t i = 0; i < poly.p.size(); i++)
    {
        const Point a = poly.p[i], b = poly.p[poly.nxt(i)];
        ans += cal(circ, a, b);
    }
    return ans;
}


// 半平面交
// 排序增量法,复杂度 O(nlogn)
// 输入与返回值都是用直线表示的半平面集合
vector<Line> halfinter(vector<Line> l, const point_t lim = 1e9)
{
    const auto check = [](const Line & a, const Line & b, const Line & c) {return a.toleft(b.inter(c)) < 0;};
    // 无精度误差的方法,但注意取值范围会扩大到三次方
    /*const auto check=[](const Line &a,const Line &b,const Line &c)
    {
        const Point p=a.v*(b.v^c.v),q=b.p*(b.v^c.v)+b.v*(c.v^(b.p-c.p))-a.p*(b.v^c.v);
        return p.toleft(q)<0;
    };*/
    l.push_back({{ -lim, 0}, {0, -1}}); l.push_back({{0, -lim}, {1, 0}});
    l.push_back({{lim, 0}, {0, 1}}); l.push_back({{0, lim}, { -1, 0}});
    sort(l.begin(), l.end());
    deque<Line> q;
    for (size_t i = 0; i < l.size(); i++)
    {
        if (i > 0 && l[i - 1].v.toleft(l[i].v) == 0 && l[i - 1].v * l[i].v > eps) continue;
        while (q.size() > 1 && check(l[i], q.back(), q[q.size() - 2])) q.pop_back();
        while (q.size() > 1 && check(l[i], q[0], q[1])) q.pop_front();
        if (!q.empty() && q.back().v.toleft(l[i].v) <= 0) return vector<Line>();
        q.push_back(l[i]);
    }
    while (q.size() > 1 && check(q[0], q.back(), q[q.size() - 2])) q.pop_back();
    while (q.size() > 1 && check(q.back(), q[0], q[1])) q.pop_front();
    return vector<Line>(q.begin(), q.end());
}

// 点集形成的最小最大三角形
// 极角序扫描线,复杂度 O(n^2logn)
// 最大三角形问题可以使用凸包与旋转卡壳做到 O(n^2)
pair<point_t, point_t> minmax_triangle(const vector<Point> &vec)
{
    if (vec.size() <= 2) return {0, 0};
    vector<pair<int, int>> evt;
    evt.reserve(vec.size()*vec.size());
    point_t maxans = 0, minans = INF;
    for (size_t i = 0; i < vec.size(); i++)
    {
        for (size_t j = 0; j < vec.size(); j++)
        {
            if (i == j) continue;
            if (vec[i] == vec[j]) minans = 0;
            else evt.push_back({i, j});
        }
    }
    sort(evt.begin(), evt.end(), [&](const pair<int, int> &u, const pair<int, int> &v)
    {
        const Point du = vec[u.second] - vec[u.first], dv = vec[v.second] - vec[v.first];
        return argcmp()({du.y, -du.x}, {dv.y, -dv.x});
    });
    vector<size_t> vx(vec.size()), pos(vec.size());
    for (size_t i = 0; i < vec.size(); i++) vx[i] = i;
    sort(vx.begin(), vx.end(), [&](int x, int y) {return vec[x] < vec[y];});
    for (size_t i = 0; i < vx.size(); i++) pos[vx[i]] = i;
    for (auto [u, v] : evt)
    {
        const size_t i = pos[u], j = pos[v];
        const size_t l = min(i, j), r = max(i, j);
        const Point vecu = vec[u], vecv = vec[v];
        if (l > 0) minans = min(minans, abs((vec[vx[l - 1]] - vecu) ^ (vec[vx[l - 1]] - vecv)));
        if (r < vx.size() - 1) minans = min(minans, abs((vec[vx[r + 1]] - vecu) ^ (vec[vx[r + 1]] - vecv)));
        maxans = max({maxans, abs((vec[vx[0]] - vecu) ^ (vec[vx[0]] - vecv)), abs((vec[vx.back()] - vecu) ^ (vec[vx.back()] - vecv))});
        if (i < j) swap(vx[i], vx[j]), pos[u] = j, pos[v] = i;
    }
    return {minans, maxans};
}

// 平面最近点对
// 扫描线,复杂度 O(nlogn)
point_t closest_pair(vector<Point> points)
{
    sort(points.begin(), points.end());
    const auto cmpy = [](const Point & a, const Point & b) {if (abs(a.y - b.y) <= eps) return a.x < b.x - eps; return a.y < b.y - eps;};
    multiset<Point, decltype(cmpy)> s{cmpy};
    point_t ans = INF;
    for (size_t i = 0, l = 0; i < points.size(); i++)
    {
        const point_t sqans = sqrtl(ans) + 1; // 整数情况
        // const point_t sqans=sqrtl(ans)+1;  // 浮点数情况
        while (l < i && points[i].x - points[l].x >= sqans) s.erase(s.find(points[l++]));
        for (auto it = s.lower_bound(Point{ -INF, points[i].y - sqans}); it != s.end() && it->y - points[i].y <= sqans; it++)
        {
            ans = min(ans, points[i].dis2(*it));
        }
        s.insert(points[i]);
    }
    return ans;
}

// 判断多条线段是否有交点
// 扫描线,复杂度 O(nlogn)
bool segs_inter(const vector<Segment> &segs)
{
    if (segs.empty()) return false;
    using seq_t = tuple<point_t, int, Segment>;
    const auto seqcmp = [](const seq_t & u, const seq_t & v)
    {
        const auto [u0, u1, u2] = u;
        const auto [v0, v1, v2] = v;
        if (abs(u0 - v0) <= eps) return make_pair(u1, u2) < make_pair(v1, v2);
        return u0 < v0 - eps;
    };
    vector<seq_t> seq;
    for (auto seg : segs)
    {
        if (seg.a.x > seg.b.x + eps) swap(seg.a, seg.b);
        seq.push_back({seg.a.x, 0, seg});
        seq.push_back({seg.b.x, 1, seg});
    }
    sort(seq.begin(), seq.end(), seqcmp);
    point_t x_now;
    auto cmp = [&](const Segment & u, const Segment & v)
    {
        if (abs(u.a.x - u.b.x) <= eps || abs(v.a.x - v.b.x) <= eps) return u.a.y < v.a.y - eps;
        return ((x_now - u.a.x) * (u.b.y - u.a.y) + u.a.y * (u.b.x - u.a.x)) * (v.b.x - v.a.x) < ((x_now - v.a.x) * (v.b.y - v.a.y) + v.a.y * (v.b.x - v.a.x)) * (u.b.x - u.a.x) - eps;
    };
    multiset<Segment, decltype(cmp)> s{cmp};
    for (const auto [x, o, seg] : seq)
    {
        x_now = x;
        const auto it = s.lower_bound(seg);
        if (o == 0)
        {
            if (it != s.end() && seg.is_inter(*it)) return true;
            if (it != s.begin() && seg.is_inter(*prev(it))) return true;
            s.insert(seg);
        }
        else
        {
            if (next(it) != s.end() && it != s.begin() && (*prev(it)).is_inter(*next(it))) return true;
            s.erase(it);
        }
    }
    return false;
}

// 多边形面积并
// 轮廓积分,复杂度 O(n^2logn),n为边数
// ans[i] 表示被至少覆盖了 i+1 次的区域的面积
vector<long double> area_union(const vector<Polygon> &polys)
{
    const size_t siz = polys.size();
    vector<vector<pair<Point, Point>>> segs(siz);
    const auto check = [](const Point & u, const Segment & e) {return !((u < e.a && u < e.b) || (u > e.a && u > e.b));};

    auto cut_edge = [&](const Segment & e, const size_t i)
    {
        const Line le{e.a, e.b - e.a};
        vector<pair<Point, int>> evt;
        evt.push_back({e.a, 0}); evt.push_back({e.b, 0});
        for (size_t j = 0; j < polys.size(); j++)
        {
            if (i == j) continue;
            const auto &pj = polys[j];
            for (size_t k = 0; k < pj.p.size(); k++)
            {
                const Segment s = {pj.p[k], pj.p[pj.nxt(k)]};
                if (le.toleft(s.a) == 0 && le.toleft(s.b) == 0)
                {
                    evt.push_back({s.a, 0});
                    evt.push_back({s.b, 0});
                }
                else if (s.is_inter(le))
                {
                    const Line ls{s.a, s.b - s.a};
                    const Point u = le.inter(ls);
                    if (le.toleft(s.a) < 0 && le.toleft(s.b) >= 0) evt.push_back({u, -1});
                    else if (le.toleft(s.a) >= 0 && le.toleft(s.b) < 0) evt.push_back({u, 1});
                }
            }
        }
        sort(evt.begin(), evt.end());
        if (e.a > e.b) reverse(evt.begin(), evt.end());
        int sum = 0;
        for (size_t i = 0; i < evt.size(); i++)
        {
            sum += evt[i].second;
            const Point u = evt[i].first, v = evt[i + 1].first;
            if (!(u == v) && check(u, e) && check(v, e)) segs[sum].push_back({u, v});
            if (v == e.b) break;
        }
    };

    for (size_t i = 0; i < polys.size(); i++)
    {
        const auto &pi = polys[i];
        for (size_t k = 0; k < pi.p.size(); k++)
        {
            const Segment ei = {pi.p[k], pi.p[pi.nxt(k)]};
            cut_edge(ei, i);
        }
    }
    vector<long double> ans(siz);
    for (size_t i = 0; i < siz; i++)
    {
        long double sum = 0;
        sort(segs[i].begin(), segs[i].end());
        int cnt = 0;
        for (size_t j = 0; j < segs[i].size(); j++)
        {
            if (j > 0 && segs[i][j] == segs[i][j - 1]) segs[i + (++cnt)].push_back(segs[i][j]);
            else cnt = 0, sum += segs[i][j].first ^ segs[i][j].second;
        }
        ans[i] = sum / 2;
    }
    return ans;
}

// 圆面积并
// 轮廓积分,复杂度 O(n^2logn)
// ans[i] 表示被至少覆盖了 i+1 次的区域的面积
vector<long double> area_union(const vector<Circle> &circs)
{
    const size_t siz = circs.size();
    using arc_t = tuple<Point, long double, long double, long double>;
    vector<vector<arc_t>> arcs(siz);
    const auto eq = [](const arc_t & u, const arc_t & v)
    {
        const auto [u1, u2, u3, u4] = u;
        const auto [v1, v2, v3, v4] = v;
        return u1 == v1 && abs(u2 - v2) <= eps && abs(u3 - v3) <= eps && abs(u4 - v4) <= eps;
    };

    auto cut_circ = [&](const Circle & ci, const size_t i)
    {
        vector<pair<long double, int>> evt;
        evt.push_back({ -PI, 0}); evt.push_back({PI, 0});
        int init = 0;
        for (size_t j = 0; j < circs.size(); j++)
        {
            if (i == j) continue;
            const Circle &cj = circs[j];
            if (ci.r < cj.r - eps && ci.relation(cj) >= 3) init++;
            const auto inters = ci.inter(cj);
            if (inters.size() == 1) evt.push_back({atan2l((inters[0] - ci.c).y, (inters[0] - ci.c).x), 0});
            if (inters.size() == 2)
            {
                const Point dl = inters[0] - ci.c, dr = inters[1] - ci.c;
                long double argl = atan2l(dl.y, dl.x), argr = atan2l(dr.y, dr.x);
                if (abs(argl + PI) <= eps) argl = PI;
                if (abs(argr + PI) <= eps) argr = PI;
                if (argl > argr + eps)
                {
                    evt.push_back({argl, 1}); evt.push_back({PI, -1});
                    evt.push_back({ -PI, 1}); evt.push_back({argr, -1});
                }
                else
                {
                    evt.push_back({argl, 1});
                    evt.push_back({argr, -1});
                }
            }
        }
        sort(evt.begin(), evt.end());
        int sum = init;
        for (size_t i = 0; i < evt.size(); i++)
        {
            sum += evt[i].second;
            if (abs(evt[i].first - evt[i + 1].first) > eps) arcs[sum].push_back({ci.c, ci.r, evt[i].first, evt[i + 1].first});
            if (abs(evt[i + 1].first - PI) <= eps) break;
        }
    };

    const auto oint = [](const arc_t & arc)
    {
        const auto [cc, cr, l, r] = arc;
        if (abs(r - l - PI - PI) <= eps) return 2.0l * PI * cr * cr;
        return cr * cr * (r - l) + cc.x * cr * (sin(r) - sin(l)) - cc.y * cr * (cos(r) - cos(l));
    };

    for (size_t i = 0; i < circs.size(); i++)
    {
        const auto &ci = circs[i];
        cut_circ(ci, i);
    }
    vector<long double> ans(siz);
    for (size_t i = 0; i < siz; i++)
    {
        long double sum = 0;
        sort(arcs[i].begin(), arcs[i].end());
        int cnt = 0;
        for (size_t j = 0; j < arcs[i].size(); j++)
        {
            if (j > 0 && eq(arcs[i][j], arcs[i][j - 1])) arcs[i + (++cnt)].push_back(arcs[i][j]);
            else cnt = 0, sum += oint(arcs[i][j]);
        }
        ans[i] = sum / 2;
    }
    return ans;
}\end{lstlisting}

\end{document}